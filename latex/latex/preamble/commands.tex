%%%%%%%%%%%%%%%%%%%%%%%%%%%%%%%%%%%%%%%%%%%%%%%%%%%%%%%%%%%%%%%%%%%%%%%%%%%%%%%
%                                                                             %
%               commands.tex - eigene Befehle für diverse Dinge               %
%                                                                             %
%%%%%%%%%%%%%%%%%%%%%%%%%%%%%%%%%%%%%%%%%%%%%%%%%%%%%%%%%%%%%%%%%%%%%%%%%%%%%%%

%-----------------------------------------------------------------------------%
%                                                   Grundlegende Definitionen %
%-----------------------------------------------------------------------------%
\author{Vorname Nachname}
\newcommand{\daAutor}{Paul Hornig, Henning Diederich und Alexander Lahn}
\newcommand{\daMatrikelnummer}{xxx, xxx, 732259}
\newcommand{\daTitel}{Ein besonderes Thema-das-interessant-ist}
\newcommand{\daTitelEins}{Verbesserung der RCA in Microservices:}
\newcommand{\daTitelZwei}{Integration eines Sprachmodells in Nezha zur}
\newcommand{\daTitelDrei}{präziseren Fehleridentifikation}
\newcommand{\daUnternehmen}{Unternehmen GmbH \& Co. KG}

\newcommand{\daUniURL}{http://www.fom.de}
\newcommand{\daUniLogo}[1]{\includegraphics[#1]{./bilder/FOM}}
\newcommand{\daStudiengang}{Bachelor Wirtschaftsinformatik}

\newcommand{\daAutorAdresse}{Musterstraße 123}
\newcommand{\daAutorPLZ}{12345}
\newcommand{\daAutorOrt}{Musterstadt}

\newcommand{\daDate}{1.2.2021}


\newcommand{\daGutachterEins}{Prof. Dr. Sascha Schlauberger}

%-----------------------------------------------------------------------------%
%                                                   einige Mathe-Definitionen %
%-----------------------------------------------------------------------------%
\newcommand{\CoG}{\operatorname{CoG}}

%-----------------------------------------------------------------------------%
%                                                  Befehl für das BibTeX-Logo %
%-----------------------------------------------------------------------------%
\makeatletter
\DeclareRobustCommand{\BibTeX}{B\hbox{\check@mathfonts
\fontsize\sf@size\z@
\math@fontsfalse\selectfont
IB}\kern-.08em
\TeX}
\makeatother

%-----------------------------------------------------------------------------%
%      Punktereihe mit größerem Abstand zwischen den Punkten als bei \dotfill %
%-----------------------------------------------------------------------------%
\makeatletter
\def\punktfill{\cleaders\hbox{$\m@th \mkern4mu . \mkern4mu$}\hfill}
\makeatother

%-----------------------------------------------------------------------------%
%                                    Makros zur Festlegung von PDF-Attributen %
%-----------------------------------------------------------------------------%
\newcommand{\setPDFCreationDate}{%               CreationDate ist Abgabedatum %
\pdfinfo{%
/CreationDate (D:%
\thedateyear%
\ifnum\value{datemonth}<10 0\fi%
\thedatemonth%
\ifnum\value{dateday}<10 0\fi%
\thedateday%
010000+01'00')}%
}
%-----------------------------------------------------------------------------%
\newcommand{\setPDFModDateToCreationDate}{% ModDate ist ebenfalls Abgabedatum %
\pdfinfo{/CreationDate (D:\thedateyear%
\ifnum\value{datemonth}<10 0\fi\thedatemonth%
\ifnum\value{dateday}<10 0\fi\thedateday010000+01'00')}}
%-----------------------------------------------------------------------------%


\gdef\twodigits#1{\ifnum#1<10 0\fi\the#1}
%-----------------------------------------------------------------------------%
\newcommand{\setPDFModDateToCurrentDate}{%        ModDate ist aktuelles Datum
\begingroup
\count0=\time \divide\count0 by 60
\edef\x{\twodigits{\count0}}%
\multiply\count0 by 60
\count1=\time \advance\count1 by -\count0
\edef\x{\x\twodigits{\count1}}%
\edef\x{/ModDate (D:\the\year 
\twodigits\month \twodigits\day \x 00+01'00')}%
\expandafter\endgroup
\expandafter\pdfinfo\expandafter{\x}%
}

%-----------------------------------------------------------------------------%
%                  Titel, Thema, Author, Schlüsselwörter und Erzeugerprogramm %
%-----------------------------------------------------------------------------%
\newcommand{\daHyperSetup}{%
\hypersetup{%
pdftitle    = {\protect{\daTitel}},%
pdfsubject  = {Diplomarbeit an der Universität Paderborn},%
pdfauthor   = {\protect{\daAutor}},%
pdfkeywords = {\daPDFKeywords},%
pdfcreator = {LaTeX2e mit {\frqq KOMA-Script\flqq} Book-Klasse und {\frqq hyperref\flqq}-Package}}
}
%-----------------------------------------------------------------------------%
%          Befehle für bestimmte deutsche Abkürzungen mit passenden Abständen %
%                                         (kein Anspruch auf Vollständigkeit) %
%-----------------------------------------------------------------------------%
\newlength{\abkz}
\setlength{\abkz}{1pt}
\newcommand{\abkrz}{\hspace{\abkz}}

\newcommand{\bzw}{bzw.\@\xspace}
\newcommand{\bzgl}{bezüglich\@\xspace}
\newcommand{\ca}{ca.\@\xspace}
\newcommand{\dah}{\mbox{d.{\abkrz}h.}\@\xspace}
\newcommand{\Dah}{D.{\abkrz}h.\@\xspace}
\newcommand{\ds}{d.{\abkrz}s.\@\xspace}
\newcommand{\evtl}{evtl.\@\xspace}
\newcommand{\etc}{etc.\@\xspace}
\newcommand{\ua}{u.{\abkrz}a.\@\xspace}
\newcommand{\Ua}{U.{\abkrz}a.\@\xspace}
\newcommand{\usw}{usw.\@\xspace}
\newcommand{\va}{v.{\abkrz}a.\@\xspace}
\newcommand{\vgl}{vgl.\@\xspace}
\newcommand{\zB}{\mbox{z.{\abkrz}B.}\@\xspace}
\newcommand{\ZB}{Zum Beispiel\xspace}
\newcommand{\zT}{z.{\abkrz}T.\@\xspace}
\newcommand{\ZT}{Zum Teil\xspace}
\newcommand{\zZ}{zur Zeit\xspace}
\newcommand{\sa}{s.{\abkrz}a.\@\xspace}
\newcommand{\ia}{i.{\abkrz}a.\@\xspace}
\newcommand{\uU}{u.{\abkrz}U.\@\xspace}
\newcommand{\bspw}{beispielsweise\@\xspace}
\newcommand{\Bspw}{Beispielsweise\@\xspace}
\newcommand{\ggf}{gegebenenfalls\@\xspace}
\newcommand{\sog}{sogenannte\@\xspace}
\newcommand{\soger}{sogenannter\@\xspace}
\newcommand{\sogen}{sogenannten\@\xspace}
\newcommand{\soges}{sogenanntes\@\xspace}
\newcommand{\og}{o.\,g.\@\xspace}

%-----------------------------------------------------------------------------%
%        erzeugt eingeklammerten, kursiven Begriff mit dem Wort "engl." davor %
%-----------------------------------------------------------------------------%
\newcommand{\engl}[1]{%
(engl. \emph{#1})\@\xspace}%

%-----------------------------------------------------------------------------%
%                    Befehle für das Verzeichnis der Akronyme und Abkürzungen %
%-----------------------------------------------------------------------------%
\newlength{\acronWidth}
\newcommand{\acron}[2]{%
\protect\hyperdef{acDef}{#1}{\rule{0pt}{1pt}}\\[-3ex]%
\makebox[\acronWidth][l]{#1}~~{#2}\par%
}
\newcommand{\acronShort}[1]{\protect\hyperlink{acDef.#1}{#1}}
\newcommand{\acronShortPlural}[1]{\protect\hyperlink{acDef.#1}{#1{s}}}


%-----------------------------------------------------------------------------%
%                         Neudefinition der Referenzierungs-Befehle (kleiner) %
%-----------------------------------------------------------------------------%
%\makeatletter\renewcommand{\@cite}[2]{[{\small #1\if@tempswa , #2\fi}]}\makeatother
%\makeatletter\renewcommand{\@biblabel}[1]{[{\small #1}]}\makeatother

%-----------------------------------------------------------------------------%
%   Gestaltung des Datei-Verweises zur CD-ROM, falls BibTeX-Item cdlink = {1} %
%                                         (siehe dazu die BibTeX-Style-Datei) %
%-----------------------------------------------------------------------------%
\newcommand{\CDLink}[1]{%
~\mbox{$\rightarrow$ {\itshape\footnotesize %
Auf beiliegender CD:}~~\href{file:./literatur/#1}{%
\raisebox{-0.5pt}{\includegraphics[height=7.5pt]{cdIconBlue}}%
\textcolor{diplaBlue}{\ttfamily\footnotesize/literatur/#1.pdf}}}}

%-----------------------------------------------------------------------------%
\newcommand{\blindtext}{%                                eigener Blindtext ;) %
%-----------------------------------------------------------------------------%
Weit hinten, hinter den Wortbergen, fern der Länder Vokalien und 
Konsonantien leben die Blindtexte. Abgeschieden wohnen sie in 
Buchstabhausen an der Küste des Semantik, eines großen Sprachozeans. 
Ein kleines Bächlein namens Duden fließt durch ihren Ort und versorgt 
sie mit den nötigen Regelialien. Es ist ein paradiesmatisches Land, 
in dem einem gebratene Satzteile in den Mund fliegen. Nicht einmal 
von der allmächtigen Interpunktion werden die Blindtexte 
beherrscht -- ein geradezu unorthographisches Leben. 

Eines Tages aber beschloß eine kleine Zeile Blindtext, 
ihr Name war Lorem Ipsum, hinaus zu gehen in die weite Grammatik. 
Der große Oxmox riet ihr davon ab, da es dort wimmele von bösen Kommata, 
wilden Fragezeichen und hinterhältigen Semikola, doch das Blindtextchen 
ließ sich nicht beirren. Es packte seine sieben Versalien, schob sich 
sein Initial in den Gürtel und machte sich auf den Weg. 
Als es die ersten Hügel des Kursivgebirges erklommen hatte, warf es 
einen letzten Blick zurück auf die Skyline seiner Heimatstadt Buchstabhausen, 
die Headline von Alphabetdorf und die Subline seiner eigenen Straße, 
der Zeilengasse. Wehmütig lief ihm eine rhetorische Frage über die Wange, 
dann setzte es seinen Weg fort. Unterwegs traf es eine Copy. Die Copy 
warnte das Blindtextchen, da, wo sie her käme wäre sie zigmal umgeschrieben 
worden und alles, was von ihrem Ursprung noch übrig wäre, sei das Wort "`und"' 
und das Blindtextchen solle umkehren und wieder in sein eigenes, sicheres Land 
zurückkehren. Doch alles Gut zureden konnte es nicht überzeugen und so dauerte 
es nicht lange, bis ihm ein paar heimtückische Werbetexter auflauerten, es mit 
Langue und Parole betrunken machten und es dann in ihre Agentur schleppten, 
wo sie es für ihre Projekte wieder und wieder mißbrauchten. Und wenn es nicht 
umgeschrieben wurde, dann benutzen sie es immer noch.%
}



%=============================================================================%
%                                 Befehle zum Setzen eines Verzeichnis-Baumes %
%=============================================================================%
\newcounter{dirCounter}
\newcommand{\cdIcon}[1]{%
\raisebox{-2pt}{\includegraphics[width=10pt]{cdIcon}}~%
{\ttfamily\bfseries\itshape\small #1}\@\xspace
}
%-----------------------------------------------------------------------------%
%                                     Befehle für Ordner/Dateien auf 1. Ebene %
%-----------------------------------------------------------------------------%
%                                offener Order mit mit Folgemuster:  E        %
%-----------------------------------------------------------------------------%
\newcommand{\oEm}[2][21]{%
\begin{picture}(12,6)
  \linethickness{.6pt}
  \put(5,-6){\line(0,1){#1}}
  \put(5,3){\line(1,0){5}}
  \put(5,3){\makebox(0,0){\includegraphics[width=5pt]{minus}}}
\end{picture}%
\raisebox{-0.5pt}{\includegraphics[width=12pt]{oOrdner}}~%
{\ttfamily\bfseries\itshape\footnotesize #2}\@\xspace
}
%-----------------------------------------------------------------------------%
%                          geschlossener Order mit mit Folgemuster:  E        %
%-----------------------------------------------------------------------------%
\newcommand{\oEp}[2][21]{%
\begin{picture}(12,6)
  \linethickness{.6pt}
  \put(5,-5){\line(0,1){#1}}
  \put(5,3){\line(1,0){5}}
  \put(5,3){\makebox(0,0){\includegraphics[width=5pt]{plus}}}
\end{picture}%
\raisebox{-0.5pt}{\includegraphics[width=12pt]{ordner}}~%
{\ttfamily\bfseries\itshape\footnotesize #2}\@\xspace
}
%-----------------------------------------------------------------------------%
%                                    offener Order mit Folgemuster:  L        %
%-----------------------------------------------------------------------------%
\newcommand{\oLm}[2][13]{%
\begin{picture}(12,6)
  \linethickness{.6pt}
  \put(5,3){\line(0,1){#1}}
  \put(5,3){\line(1,0){5}}
  \put(5,3){\makebox(0,0){\includegraphics[width=5pt]{minus}}}
\end{picture}%
\raisebox{-0.5pt}{\includegraphics[width=12pt]{oOrdner}}~%
{\ttfamily\bfseries\itshape\footnotesize #2}\@\xspace
}
%-----------------------------------------------------------------------------%
%                              geschlossener Order mit Folgemuster:  L        %
%-----------------------------------------------------------------------------%
\newcommand{\oLp}[2][13]{%
\begin{picture}(12,6)
  \linethickness{.6pt}
  \put(5,3){\line(0,1){#1}}
  \put(5,3){\line(1,0){5}}
  \put(5,3){\makebox(0,0){\includegraphics[width=5pt]{plus}}}
\end{picture}%
\raisebox{-0.5pt}{\includegraphics[width=12pt]{ordner}}~%
{\ttfamily\bfseries\itshape\footnotesize #2}\@\xspace
}
%-----------------------------------------------------------------------------%
%                                                          Datei auf 1. Ebene %
%-----------------------------------------------------------------------------%
\newcommand{\dI}[3][21]{%
\begin{picture}(12,6)
  \linethickness{.6pt}
  \put(5,-5){\line(0,1){#1}}
\end{picture}{\ttfamily\footnotesize #2}~\punktfill~{\small #3}}
%-----------------------------------------------------------------------------%

%-----------------------------------------------------------------------------%
%                                 Befehle für Ordner/Dateien auf der 2. Ebene %
%-----------------------------------------------------------------------------%
%                                    offener Order mit Folgemuster:  I  E     %
%-----------------------------------------------------------------------------%
\newcommand{\oIEm}[2][21]{%
\begin{picture}(24,6)
  \linethickness{.6pt}
  \put(5,-5){\line(0,1){#1}}
  \put(17,-5){\line(0,1){#1}}
  \put(17,3){\line(1,0){5}}
  \put(17,3){\makebox(0,0){\includegraphics[width=5pt]{minus}}}
  \end{picture}%
\raisebox{-0.5pt}{\includegraphics[width=12pt]{oOrdner}}~%
{\ttfamily\bfseries\itshape\footnotesize #2}\@\xspace
}
%-----------------------------------------------------------------------------%
%                              geschlossener Order mit Folgemuster:  I  E     %
%-----------------------------------------------------------------------------%
\newcommand{\oIEp}[2][21]{%
\begin{picture}(24,6)
  \linethickness{.6pt}
  \put(5,-5){\line(0,1){#1}}
  \put(17,-5){\line(0,1){#1}}
  \put(17,3){\line(1,0){5}}
  \put(17,3){\makebox(0,0){\includegraphics[width=5pt]{plus}}}
  \end{picture}%
\raisebox{-0.5pt}{\includegraphics[width=12pt]{ordner}}~%
{\ttfamily\bfseries\itshape\footnotesize #2}\@\xspace
}
%-----------------------------------------------------------------------------%
%                                    offener Order mit Folgemuster:  I  L     %
%-----------------------------------------------------------------------------%
\newcommand{\oILm}[2][21]{%
\setcounter{dirCounter}{#1}
\addtocounter{dirCounter}{-8}
\begin{picture}(24,6)
  \linethickness{.6pt}
  \put(5,-5){\line(0,1){#1}}
  \put(17,3){\line(0,1){\value{dirCounter}}}
  \put(17,3){\line(1,0){5}}
  \put(17,3){\makebox(0,0){\includegraphics[width=5pt]{minus}}}
\end{picture}%
\raisebox{-0.5pt}{\includegraphics[width=12pt]{oOrdner}}~%
{\ttfamily\bfseries\itshape\footnotesize #2}\@\xspace
}
%-----------------------------------------------------------------------------%
%                              geschlossener Order mit Folgemuster:  I  L     %
%-----------------------------------------------------------------------------%
\newcommand{\oILp}[2][21]{%
\setcounter{dirCounter}{#1}
\addtocounter{dirCounter}{-8}
\begin{picture}(24,6)
  \linethickness{.6pt}
  \put(5,-5){\line(0,1){#1}}
  \put(17,3){\line(0,1){\value{dirCounter}}}
  \put(17,3){\line(1,0){5}}
  \put(17,3){\makebox(0,0){\includegraphics[width=5pt]{plus}}}
\end{picture}%
\raisebox{-0.5pt}{\includegraphics[width=12pt]{ordner}}~%
{\ttfamily\bfseries\itshape\footnotesize #2}\@\xspace
}
%-----------------------------------------------------------------------------%
%                                    offener Order mit Folgemuster:     E     %
%-----------------------------------------------------------------------------%
\newcommand{\oXEm}[2][21]{%
\begin{picture}(24,6)
  \linethickness{.6pt}
  \put(17,-5){\line(0,1){#1}}
  \put(17,3){\line(1,0){5}}
  \put(17,3){\makebox(0,0){\includegraphics[width=5pt]{minus}}}
  \end{picture}%
\raisebox{-0.5pt}{\includegraphics[width=12pt]{oOrdner}}~%
{\ttfamily\bfseries\itshape\footnotesize #2}\@\xspace
}
%-----------------------------------------------------------------------------%
%                              geschlossener Order mit Folgemuster:     E     %
%-----------------------------------------------------------------------------%
\newcommand{\oXEp}[2][21]{%
\begin{picture}(24,6)
  \linethickness{.6pt}
  \put(17,-5){\line(0,1){#1}}
  \put(17,3){\line(1,0){5}}
  \put(17,3){\makebox(0,0){\includegraphics[width=5pt]{plus}}}
  \end{picture}%
\raisebox{-0.5pt}{\includegraphics[width=12pt]{ordner}}~%
{\ttfamily\bfseries\itshape\footnotesize #2}\@\xspace
}
%-----------------------------------------------------------------------------%
%                                    offener Order mit Folgemuster:     L     %
%-----------------------------------------------------------------------------%
\newcommand{\oXLm}[2][13]{%
\begin{picture}(24,6)
  \linethickness{.6pt}
  \put(17,3){\line(0,1){#1}}
  \put(17,3){\line(1,0){5}}
  \put(17,3){\makebox(0,0){\includegraphics[width=5pt]{minus}}}
\end{picture}%
\raisebox{-0.5pt}{\includegraphics[width=12pt]{oOrdner}}~%
{\ttfamily\bfseries\itshape\footnotesize #2}\@\xspace
}
%-----------------------------------------------------------------------------%
%                              geschlossener Order mit Folgemuster:     L     %
%-----------------------------------------------------------------------------%
\newcommand{\oXLp}[2][13]{%
\begin{picture}(24,6)
  \linethickness{.6pt}
  \put(17,3){\line(0,1){#1}}
  \put(17,3){\line(1,0){5}}
  \put(17,3){\makebox(0,0){\includegraphics[width=5pt]{plus}}}
\end{picture}%
\raisebox{-0.5pt}{\includegraphics[width=12pt]{ordner}}~%
{\ttfamily\bfseries\itshape\footnotesize #2}\@\xspace
}
%-----------------------------------------------------------------------------%
%                                            Datei mit Folgemuster:  I        %
%-----------------------------------------------------------------------------%
\newcommand{\dII}[3][23]{%
\begin{picture}(23,6)
  \linethickness{.6pt}
  \put(5,-5){\line(0,1){#1}}
  \put(17,-5){\line(0,1){#1}}
\end{picture}~%
\texttt{\footnotesize #2}~\punktfill~{\small #3}}
%-----------------------------------------------------------------------------%
%                                            Datei mit Folgemuster:           %
%-----------------------------------------------------------------------------%
\newcommand{\dXI}[3][23]{%
\begin{picture}(23,6)
  \linethickness{.6pt}
  \put(17,-5){\line(0,1){#1}}
\end{picture}~%
\texttt{\footnotesize #2}~\punktfill~{\small #3}}
%-----------------------------------------------------------------------------%
%                                 Befehle für Ordner/Dateien auf der 3. Ebene %
%-----------------------------------------------------------------------------%
%                                    offener Order mit Folgemuster:  I  I  L  %
%-----------------------------------------------------------------------------%
\newcommand{\oIILm}[2][21]{%
\setcounter{dirCounter}{#1}
\addtocounter{dirCounter}{-8}
\begin{picture}(36,6)
  \linethickness{.6pt}
  \put(5,-5){\line(0,1){#1}}
  \put(17,-5){\line(0,1){#1}}
  \put(29,3){\line(0,1){\value{dirCounter}}}
  \put(29,3){\line(1,0){5}}
  \put(29,3){\makebox(0,0){\includegraphics[width=5pt]{minus}}}
\end{picture}%
\raisebox{-0.5pt}{\includegraphics[width=12pt]{oOrdner}}~%
{\ttfamily\bfseries\itshape\footnotesize #2}\@\xspace
}
%-----------------------------------------------------------------------------%
%                              geschlossener Order mit Folgemuster:  I  I  L  %
%-----------------------------------------------------------------------------%
\newcommand{\oIILp}[2][21]{%
\setcounter{dirCounter}{#1}
\addtocounter{dirCounter}{-8}
\begin{picture}(36,6)
  \linethickness{.6pt}
  \put(5,-5){\line(0,1){#1}}
  \put(17,-5){\line(0,1){#1}}
  \put(29,3){\line(0,1){\value{dirCounter}}}
  \put(29,3){\line(1,0){5}}
  \put(29,3){\makebox(0,0){\includegraphics[width=5pt]{plus}}}
\end{picture}%
\raisebox{-0.5pt}{\includegraphics[width=12pt]{ordner}}~%
{\ttfamily\bfseries\itshape\footnotesize #2}\@\xspace
}
%-----------------------------------------------------------------------------%
%                                    offener Order mit Folgemuster:  I  I  E  %
%-----------------------------------------------------------------------------%
\newcommand{\oIIE}[2][21]{%
\begin{picture}(36,6)
  \linethickness{.6pt}
  \put(5,-5){\line(0,1){#1}}
  \put(17,-5){\line(0,1){#1}}
  \put(29,-5){\line(0,1){#1}}
  \put(29,3){\line(1,0){5}}
  \put(29,3){\makebox(0,0){\includegraphics[width=5pt]{minus}}}
\end{picture}%
\raisebox{-0.5pt}{\includegraphics[width=10pt]{ordner}}~%
{\ttfamily\bfseries\itshape\footnotesize #2}\@\xspace
}
%-----------------------------------------------------------------------------%
%                              geschlossener Order mit Folgemuster:  I  I  E  %
%-----------------------------------------------------------------------------%
\newcommand{\oIIEp}[2][21]{%
\begin{picture}(36,6)
  \linethickness{.6pt}
  \put(5,-5){\line(0,1){#1}}
  \put(17,-5){\line(0,1){#1}}
  \put(29,-5){\line(0,1){#1}}
  \put(29,3){\line(1,0){5}}
  \put(29,3){\makebox(0,0){\includegraphics[width=5pt]{plus}}}
\end{picture}%
\raisebox{-0.5pt}{\includegraphics[width=12pt]{ordner}}~%
{\ttfamily\bfseries\itshape\footnotesize #2}\@\xspace
}
%-----------------------------------------------------------------------------%
%                              geschlossener Order mit Folgemuster:  I     L  %
%-----------------------------------------------------------------------------%
\newcommand{\oIXLp}[2][21]{%
\setcounter{dirCounter}{#1}
\addtocounter{dirCounter}{-8}
\begin{picture}(36,6)
  \linethickness{.6pt}
  \put(5,-5){\line(0,1){#1}}
  \put(29,3){\line(0,1){\value{dirCounter}}}
  \put(29,3){\line(1,0){5}}
  \put(29,3){\makebox(0,0){\includegraphics[width=5pt]{plus}}}
\end{picture}%
\raisebox{-0.5pt}{\includegraphics[width=10pt]{ordner}}~%
{\ttfamily\bfseries\itshape\footnotesize #2}\@\xspace
}
%-----------------------------------------------------------------------------%
%                                    offener Order mit Folgemuster:  I     L  %
%-----------------------------------------------------------------------------%
\newcommand{\oIXLm}[2][21]{%
\setcounter{dirCounter}{#1}
\addtocounter{dirCounter}{-8}
\begin{picture}(36,6)
  \linethickness{.6pt}
  \put(5,-5){\line(0,1){#1}}
  \put(29,3){\line(0,1){\value{dirCounter}}}
  \put(29,3){\line(1,0){5}}
  \put(29,3){\makebox(0,0){\includegraphics[width=5pt]{minus}}}
\end{picture}%
\raisebox{-0.5pt}{\includegraphics[width=10pt]{oOrdner}}~%
{\ttfamily\bfseries\itshape\footnotesize #2}\@\xspace
}
%-----------------------------------------------------------------------------%
%                              geschlossener Order mit Folgemuster:  I     E  %
%-----------------------------------------------------------------------------%
\newcommand{\oIXEp}[2][21]{%
\begin{picture}(36,6)
  \linethickness{.6pt}
  \put(5,-5){\line(0,1){#1}}
  \put(29,-5){\line(0,1){#1}}
  \put(29,3){\line(1,0){5}}
  \put(29,3){\makebox(0,0){\includegraphics[width=5pt]{plus}}}
\end{picture}%
\raisebox{-0.5pt}{\includegraphics[width=12pt]{ordner}}~%
{\ttfamily\bfseries\itshape\footnotesize #2}\@\xspace
}
%-----------------------------------------------------------------------------%
%                                    offener Order mit Folgemuster:  I     E  %
%-----------------------------------------------------------------------------%
\newcommand{\oIXE}[2][21]{%
\begin{picture}(36,6)
  \linethickness{.6pt}
  \put(5,-5){\line(0,1){#1}}
  \put(29,-5){\line(0,1){#1}}
  \put(29,3){\line(1,0){5}}
  \put(29,3){\makebox(0,0){\includegraphics[width=5pt]{minus}}}
\end{picture}%
\raisebox{-0.5pt}{\includegraphics[width=10pt]{ordner}}~%
{\ttfamily\bfseries\itshape\footnotesize #2}\@\xspace
}
%------------------------------------------------------------------------------
\newcommand{\dIII}[3][21]{%
\begin{picture}(36,6)
  \linethickness{.6pt}
  \put(5,-5){\line(0,1){#1}}
  \put(17,-5){\line(0,1){#1}}
  \put(29,-5){\line(0,1){#1}}
\end{picture}{\ttfamily\footnotesize #2}~\punktfill~{\small #3}}
%------------------------------------------------------------------------------
\newcommand{\dIIx}[3][21]{%
\begin{picture}(36,6)
  \linethickness{.6pt}
  \put(5,-5){\line(0,1){#1}}
  \put(17,-5){\line(0,1){#1}}
\end{picture}{\ttfamily\footnotesize #2}~\punktfill~{\small #3}}
%------------------------------------------------------------------------------


