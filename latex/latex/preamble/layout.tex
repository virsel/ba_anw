%%%%%%%%%%%%%%%%%%%%%%%%%%%%%%%%%%%%%%%%%%%%%%%%%%%%%%%%%%%%%%%%%%%%%%%%%%%%%%%
%                                                                             %
%          layout.tex - Pakete, Layout- und Formatierungsanweisungen          %
%                                                                             %
%%%%%%%%%%%%%%%%%%%%%%%%%%%%%%%%%%%%%%%%%%%%%%%%%%%%%%%%%%%%%%%%%%%%%%%%%%%%%%%

\usepackage[utf8]{inputenc}%                       verwendetes Input Encoding %
\usepackage[TS1,T1]{fontenc}%                       verwendete Font Encodings %
\usepackage{csquotes}%                                      Anführungszeichen %
\usepackage{amsmath,amsfonts,amssymb}%   AMS Mathematik-Hilfsmittel für LaTeX %
\usepackage{esvect}%                                      Schöne Vektorpfeile %
\usepackage{textcomp}%                 Unterstützung der Text Companion Fonts %
\usepackage[pdftex]{graphicx}%           Unterstützung für Graphik-Einbindung %
\usepackage[pdftex,hyperref,cmyk,dvipsnames]{xcolor}%          Treiber-unabhängige Farb- % 
%                                        erweiterungen für LaTeX und pdfLaTeX %
% \definecolor{uniblue}{cmyk}{1 0.72 0 0.38}%       Farbdefinition des Uni-Blau %
% \definecolor{diplaBlue}{cmyk}{1 0.8 0 0}%                "       schönes Blau %

\usepackage{array}%     erweiterte Implementierung der array und tabular Umg. %
\usepackage{tabularx}%     Ausdehnung von Tabellen auf eine definierte Breite %
\usepackage{colortbl}%                     erlaubt farbige Zeilen und Spalten %
\usepackage{booktabs}%               typographisch "schöne" Tabellen in LaTeX %
\usepackage{longtable}%                                        lange Tabellen %
\usepackage{setspace}%       Festlegung von unterschiedlichen Zeilenabständen %
\usepackage{scrlayer-scrpage}%             Neuere Version der scrpage2 Klasse %
\usepackage{xspace}%                definiert Befehle "that don't eat spaces" %
\usepackage[german]{datenumber}%       Konvertierung zwischen Datum <--> Zahl %
\usepackage[ngerman]{babel}

\usepackage{xfrac} %               Typographisch korrekte Brüche im Fließtext %
\usepackage{changes}

\usepackage{ulem}% Durchstreichung %

\usepackage[%
   pdftex,%                                                     Backend Driver %
   colorlinks          = true,%                   Färbung von Links und Ankern %
  linkcolor           = black!70,%               Farbe für normale interne Links %
%   anchorcolor         = black,%                          Farbe für Anker-Text %
%   citecolor           = black,%           Farbe für Literaturhinweise im Text %
%   filecolor           = black,%             Farbe für URLs zu lokalen Dateien %
%   menucolor           = black,%               Farbe für Acrobat Menü-Einträge %
%   pagecolor           = gray,%             Farbe für Links zu anderen Seiten %
   urlcolor            = blue,%                        Farbe für verinkte URLs %
   pdfstartview        = {FitH},%         Festlegung der Startup-Seitenansicht %
   pdfpagelabels,%                                                             %
   pdfpagelayout       = {SinglePage},%           Festlegung der Layoutansicht %
   bookmarksopen       = true,%         Anzeige von Bookmarks, falls verwendet %
   bookmarksopenlevel  = 1,%      Tiefe, bis zu der Bookmarks angezeigt werden %
   bookmarksnumbered   = true,%                                                %
   breaklinks,%                               Links »überstehen« Zeilenumbruch %
   plainpages          = false,%     notwendig wg. der Seiten(neu)nummerierung %
%                        von Vorspann und Haupttext, andernfalls pdf-Warnungen %
]{hyperref}

\usepackage{hypcap}%                              Legt den Anker für pdf-link %
%                     auf den Anfang eines Bildes, nicht auf die Unterschrift %

\usepackage[
    printonlyused
]{acronym}
\usepackage[section]{placeins}


\usepackage[titles]{tocloft}%              individuelle Formatierungen am TOC %

%-----------------------------------------------------------------------------%
%              andere Schriftauswahl statt der Standardschriften, siehe dazu: %
%                                       http://home.vr-web.de/~was/fonts.html %
%------------------------------------------------------------------------------
\usepackage{mathptmx}%                 Times als Grund- und Mathematikschrift %
%\usepackage{mathpazo}%              Palatino als Grund- und Mathematikschrift %
\usepackage[scaled=0.9]{helvet}%   Helvetica (psnfss) als serifenlose Schrift %
\renewcommand{\bfdefault}{b}%
\usepackage[scaled=0.9]{beramono}%              freie Bitstream-Familie Bera, %
%                                daraus die Monospaced als Typewriter-Schrift %
%-----------------------------------------------------------------------------%
%     diverse Pakete, z.B. microtype für pdfTeXs micro-typographic extensions %
%-----------------------------------------------------------------------------%
\usepackage[stretch=15,shrink=15,step=5]{microtype}
\usepackage{dirtree, ellipsis, fixltx2e, mparhack}
%-----------------------------------------------------------------------------%
%                                Einheitliche Formatierung für Unterschriften %
%-----------------------------------------------------------------------------%
\usepackage[
	format        = hang,
	justification = RaggedRight,
	position      = below,
	width         = .9\textwidth
]{caption}
\usepackage{subcaption}%       Neueres Paket zur Unterstützung von Subfigures %

%-----------------------------------------------------------------------------%
%                                                    Quelltext-Formatierungen %
%-----------------------------------------------------------------------------%
\usepackage{listings}
\lstset{
    language=c++,         % Sourcecode language ist C++
    numbers=none,            % No linenumbers
%    stepnumber=1,            % Every line got his own number.
    numbersep=5pt,           % 5pt distance to Sourcecode
    numberstyle=\tiny,       % tiny numbers.
    breaklines=true,         % break lines if it has to do.
    breakautoindent=true,    % Nach dem Zeilenumbruch Zeile einrücken.
    postbreak=\space,        % Bei Leerzeichen umbrechen.
    tabsize=2,               % Tabulatorsize 2
    basicstyle=\ttfamily\footnotesize, % Nichtproportionale Schrift, klein für den Quellcode
    showspaces=false,        % Leerzeichen nicht anzeigen.
    showstringspaces=false,  % Leerzeichen auch in Strings ('') nicht anzeigen.
    extendedchars=true,      % Alle Zeichen vom Latin1 Zeichensatz anzeigen.
    backgroundcolor=\color{black!5}, % Hintergrundfarbe des Quellcodes setzen.
    captionpos=b,         % Caption unter und nicht über das Listing schreiben
    frame = lines,
	floatplacement = tb,
	morekeywords = {foreach},
    %rulesepcolor=\color{darkgrey}
} 
%-----------------------------------------------------------------------------%
%                                                        Schrifteinstellungen %
%------------------------------------------------------------------------------
%\linespread{1.25}\selectfont%                                    mehr Abstand %
\linespread{1.10}\selectfont%                                    mehr Abstand %

%   Redefinition Typewriter-Befehls: kleinere Schrift, Silbentrennungszeichen %
\renewcommand\texttt[1]{% 
\small\ttfamily\hyphenchar\font=\defaulthyphenchar#1{}\normalsize\rmfamily\hyphenchar\font=\defaulthyphenchar}

\setkomafont{sectioning}{%        Abschnittsüberschriften serifenlos und fett %
\sffamily\bfseries}

\addtokomafont{caption}{%                            Bildunterschriften klein %
\normalfont\normalcolor\small}

\addtokomafont{captionlabel}{\small\bfseries}%             Bezeichnungen fett %


%------------------------------------------------------------------------------
%                             individuelle Formatierungen an TOC, LOF und LOT %
%------------------------------------------------------------------------------
\renewcommand{\cftchapfont}{%
\sffamily\bfseries}%                       Kapitelbezeichnung fett & serifenlos
\renewcommand{\cftchappagefont}{\sffamily
\bfseries}%                                     Kapitelnummer fett & serifenlos
\setlength{\cftbeforesecskip}{1.5pt}%           vert. Abstand vor Abschnittsbez.
\setlength{\cftbeforesubsecskip}{1pt}%     vert. Abstand vor Unterabschnittsbez.
\cftsetindents{chapter}{0em}{1.3em}%                         Einzug für Kapitel
\cftsetindents{section}{1.3em}{2em}%                      Einzug für Abschnitte
\cftsetindents{subsection}{3.3em}{2.7em}%            Einzug für Unterabschnitte
\cftsetindents{subsubsection}{6em}{3.3em}%           Einzug für Unterabschnitte


\setlength{\cftbeforetabskip}{1.5pt}%   vert. Abstand vor Tabellenbezeichnungen
\cftsetindents{table}{0em}{2.5em}%            Einzüge für Tabellenbezeichnungen
\renewcommand{\cfttabpresnum}{\hfill}%
\renewcommand{\cfttabaftersnum}{\hspace*{0.6em}}%

\setlength{\cftbeforefigskip}{1.5pt}%             Abstand vor Bildbezeichnungen
% \setlength{\cftbeforesubfigskip}{0.5pt}%     Abstand vor Unterbildbezeichnungen
% \cftsetindents{figure}{0em}{2.5em}%               Einzüge für Bildbezeichnungen
% \cftsetindents{subfigure}{2.5em}{1.8em}%     Einzüge für Unterbildbezeichnungen
% \renewcommand{\cftfigpresnum}{\hfill}%
\renewcommand{\cftfigaftersnum}{\hspace*{0.6em}}%

%-----------------------------------------------------------------------------%
%                                 Einstellung bestimmter Längen und Parameter %
%-----------------------------------------------------------------------------%
%\parindent0mm%                     KEIN Einzug bei neuem Absatz, statt dessen %
%\parskip2mm plus1.5pt minus1.5pt%              Abstand zwischen zwei Absätzen %
%\parindent5mm%                                        Einzug bei neuem Absatz %

%                 zusätzlicher vertikaler Abstand vor und nach Gleitobjekten, %
%                                    die mit der Option "h" eingefügt werden: %
\setlength{\intextsep}{1.5\baselineskip plus 2pt minus 2pt}%

%                           zusätzlicher vertikaler Abstand für Gleitobjekte, %
%                         die mit den Optionen "t" bzw. "b" eingefügt werden: %
\setlength{\textfloatsep}{1.5\baselineskip plus 2pt minus 2pt}

%\setcapwidth[c]{.8\textwidth}%    Abbildungsbeschriftungen 80% der Textbreite %

\addtolength{\footnotesep}{2pt}%      mehr Abstand zwischen Fußnoten und Text %

\pagestyle{scrheadings}%               Seitenstil mit lebenden Kolumnentiteln %
\clearpairofpagestyles
\ohead[]{\headmark}
\cfoot[]{}
\ofoot[\pagemark]{\pagemark}
% \setheadsepline{0.08em}%    Einstellungen der Dicke der Kopfzeilen-Trennlinie %

\setcounter{secnumdepth}{3}%                               Nummerierungstiefe %
\setcounter{tocdepth}{2}%                  Inhaltsverzeichnis bis zur Tiefe 2 %
\setcounter{lofdepth}{2}%               Abbildungsverzeichnis bis zur Tiefe 2 %

\renewcommand{\topfraction}{0.9}%      Anteile der Seite, die maximal/minimal %
\renewcommand{\bottomfraction}{0.9}%    von Abbildungen bzw. Text eingenommen %
\renewcommand{\textfraction}{0.1}%                              werden können %


%------------------------------------------------------------------------------
%                  Einstellungen für den Zeilenumbruchsalgorithmus, siehe dazu:
%           --> http://www.jr-x.de/publikationen/latex/tipps/zeilenumbruch.html
%------------------------------------------------------------------------------
\emergencystretch 1.5em%
\clubpenalty = 10000%
\widowpenalty = 10000%
\displaywidowpenalty = 10000
\doublehyphendemerits = 10000%
\tolerance=500

\hfuzz 0.25pt%                     Grenze, ab der "overfull hbox" gemeldet wird
\vfuzz \hfuzz%                     Grenze, ab der "overfull vbox" gemeldet wird

%------------------------------------------------------------------------------
%                              Einstellungen des Satzspiegels mit den Vorgaben:
%                  Rand innen: 35mm, Rand außen 25 mm, Rand oben/(unten) ~25 mm
%------------------------------------------------------------------------------

%%\paperwidth    597.50787pt%                            597,50787 pt = 210,00 mm
%%\paperheight   845.04684pt%                            845,04684 pt = 297,00 mm
%%\paperwidth    210mm
%%\paperheight   297mm
%%\textwidth     426.79135pt%       210 - 35 - 25 mm  =  426,79135 pt = 150,00 mm
%\textwidth 140mm
%%\textheight    638.40076pt%        41 Zeilen           621,07825 pt = 218,28 mm
%%\oddsidemargin  27.31465pt%                             27,31465 pt =   9,60 mm
%%\evensidemargin -1.13811pt%                              1,13811 pt =  -0,40 mm
%%\oddsidemargin  10mm
%%\evensidemargin 8.5mm
%\topmargin      -8.53583pt%                              8,53583 pt =  -3,00 mm
%\headheight     17.07165pt%                             17,07165 pt =   6,00 mm
%\headsep        19.91692pt%                             19,91692 pt =   7,00 mm
%\footskip       45.52440pt%                             45,52440 pt =  16,00 mm
%\marginparwidth 59.75079pt%                             59,75079 pt =  21,00 mm
%\marginparsep    8.53583pt%                              8,53583 pt =   3,00 mm
%\skip\footins  10.8pt plus 4.0pt minus 2.0pt
%%\hoffset 0.0pt%
%%\voffset 0.0pt%

\usepackage{todonotes}

% FOM-"Standard"
% \usepackage[style=verbose-trad2, backend=biber]{biblatex}
% \counterwithout*{footnote}{chapter}
% \AtEveryCitekey{%
% \clearfield{url}%
% \clearfield{isbn}%
% \clearfield{doi}%
% }
% Stern (IEEE)
\usepackage[style=ieee, mincitenames=1, maxcitenames=2, backend=biber]{biblatex}
\usepackage{xurl}%      URLs werden so auch im Literaturverzeichnis umgebrochen %

% \citetitle: Titel mit deutschen Anführungszeichen
\DeclareFieldFormat*{citetitle}{\glqq#1\grqq}

% "et al." für die Abkürzung mehrerer Autoren verwenden
\DefineBibliographyStrings{german}{%
  andothers = {\emph{et al}\adddot}
}

% Hier die entsprechende .bib-Datei eintragen
\addbibresource{Henning.bib}
\addbibresource{Alexander.bib}
\addbibresource{Paul.bib}


% Keine Warnung für geteilte Bibliographien
\BiblatexSplitbibDefernumbersWarningOff