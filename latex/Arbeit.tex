\documentclass[%
  pdftex,%            Weiterreichen des gewählten Backend-Treibers für pdfTeX %
  a4paper,
  11pt,%                                          Größe der Grundschrift 11pt %
  oneside,%                                                zweiseitiger Druck %
  BCOR = 1cm,%                                                 Bindekorrektur %
  cleardoublepage=plain,%               Kapitelanfang immer auf rechter Seite %
  fleqn,%                                       Gleichungen links, eingerückt %
  numbers=noenddot,%                         Kapitel-Nummerierung ohne Punkte %
  headings=normal,%                                 normalgroße Überschriften %
  bibliography=totocnumbered,%                  Literaturverszeichnis ins TOC %
  listof=numbered,%               Abbildungs- und Tabellenverzeichnis ins Toc %
  ngerman,%             Weiterreichen "Neue Rechtschreibung" an weiter Pakete %
  headsepline=0.08em,%
]{scrbook}%                   DiplA aufgebaut auf der Koma-Script Buch-Klasse %

%%%%%%%%%%%%%%%%%%%%%%%%%%%%%%%%%%%%%%%%%%%%%%%%%%%%%%%%%%%%%%%%%%%%%%%%%%%%%%%
%                                                                             %
%          layout.tex - Pakete, Layout- und Formatierungsanweisungen          %
%                                                                             %
%%%%%%%%%%%%%%%%%%%%%%%%%%%%%%%%%%%%%%%%%%%%%%%%%%%%%%%%%%%%%%%%%%%%%%%%%%%%%%%

\usepackage[utf8]{inputenc}%                       verwendetes Input Encoding %
\usepackage[TS1,T1]{fontenc}%                       verwendete Font Encodings %
\usepackage{csquotes}%                                      Anführungszeichen %
\usepackage{amsmath,amsfonts,amssymb}%   AMS Mathematik-Hilfsmittel für LaTeX %
\usepackage{esvect}%                                      Schöne Vektorpfeile %
\usepackage{textcomp}%                 Unterstützung der Text Companion Fonts %
\usepackage[pdftex]{graphicx}%           Unterstützung für Graphik-Einbindung %
\usepackage[pdftex,hyperref,cmyk,dvipsnames]{xcolor}%          Treiber-unabhängige Farb- % 
%                                        erweiterungen für LaTeX und pdfLaTeX %
% \definecolor{uniblue}{cmyk}{1 0.72 0 0.38}%       Farbdefinition des Uni-Blau %
% \definecolor{diplaBlue}{cmyk}{1 0.8 0 0}%                "       schönes Blau %

\usepackage{array}%     erweiterte Implementierung der array und tabular Umg. %
\usepackage{tabularx}%     Ausdehnung von Tabellen auf eine definierte Breite %
\usepackage{colortbl}%                     erlaubt farbige Zeilen und Spalten %
\usepackage{booktabs}%               typographisch "schöne" Tabellen in LaTeX %
\usepackage{longtable}%                                        lange Tabellen %
\usepackage{setspace}%       Festlegung von unterschiedlichen Zeilenabständen %
\usepackage{scrlayer-scrpage}%             Neuere Version der scrpage2 Klasse %
\usepackage{xspace}%                definiert Befehle "that don't eat spaces" %
\usepackage[german]{datenumber}%       Konvertierung zwischen Datum <--> Zahl %
\usepackage[ngerman]{babel}

\usepackage{xfrac} %               Typographisch korrekte Brüche im Fließtext %
\usepackage{changes}

\usepackage{ulem}% Durchstreichung %

\usepackage[%
   pdftex,%                                                     Backend Driver %
   colorlinks          = true,%                   Färbung von Links und Ankern %
  linkcolor           = black!70,%               Farbe für normale interne Links %
%   anchorcolor         = black,%                          Farbe für Anker-Text %
%   citecolor           = black,%           Farbe für Literaturhinweise im Text %
%   filecolor           = black,%             Farbe für URLs zu lokalen Dateien %
%   menucolor           = black,%               Farbe für Acrobat Menü-Einträge %
%   pagecolor           = gray,%             Farbe für Links zu anderen Seiten %
   urlcolor            = blue,%                        Farbe für verinkte URLs %
   pdfstartview        = {FitH},%         Festlegung der Startup-Seitenansicht %
   pdfpagelabels,%                                                             %
   pdfpagelayout       = {SinglePage},%           Festlegung der Layoutansicht %
   bookmarksopen       = true,%         Anzeige von Bookmarks, falls verwendet %
   bookmarksopenlevel  = 1,%      Tiefe, bis zu der Bookmarks angezeigt werden %
   bookmarksnumbered   = true,%                                                %
   breaklinks,%                               Links »überstehen« Zeilenumbruch %
   plainpages          = false,%     notwendig wg. der Seiten(neu)nummerierung %
%                        von Vorspann und Haupttext, andernfalls pdf-Warnungen %
]{hyperref}

\usepackage{hypcap}%                              Legt den Anker für pdf-link %
%                     auf den Anfang eines Bildes, nicht auf die Unterschrift %

\usepackage[
    printonlyused
]{acronym}
\usepackage[section]{placeins}


\usepackage[titles]{tocloft}%              individuelle Formatierungen am TOC %

%-----------------------------------------------------------------------------%
%              andere Schriftauswahl statt der Standardschriften, siehe dazu: %
%                                       http://home.vr-web.de/~was/fonts.html %
%------------------------------------------------------------------------------
\usepackage{mathptmx}%                 Times als Grund- und Mathematikschrift %
%\usepackage{mathpazo}%              Palatino als Grund- und Mathematikschrift %
\usepackage[scaled=0.9]{helvet}%   Helvetica (psnfss) als serifenlose Schrift %
\renewcommand{\bfdefault}{b}%
\usepackage[scaled=0.9]{beramono}%              freie Bitstream-Familie Bera, %
%                                daraus die Monospaced als Typewriter-Schrift %
%-----------------------------------------------------------------------------%
%     diverse Pakete, z.B. microtype für pdfTeXs micro-typographic extensions %
%-----------------------------------------------------------------------------%
\usepackage[stretch=15,shrink=15,step=5]{microtype}
\usepackage{dirtree, ellipsis, fixltx2e, mparhack}
%-----------------------------------------------------------------------------%
%                                Einheitliche Formatierung für Unterschriften %
%-----------------------------------------------------------------------------%
\usepackage[
	format        = hang,
	justification = RaggedRight,
	position      = below,
	width         = .9\textwidth
]{caption}
\usepackage{subcaption}%       Neueres Paket zur Unterstützung von Subfigures %

%-----------------------------------------------------------------------------%
%                                                    Quelltext-Formatierungen %
%-----------------------------------------------------------------------------%
\usepackage{listings}
\lstset{
    language=c++,         % Sourcecode language ist C++
    numbers=none,            % No linenumbers
%    stepnumber=1,            % Every line got his own number.
    numbersep=5pt,           % 5pt distance to Sourcecode
    numberstyle=\tiny,       % tiny numbers.
    breaklines=true,         % break lines if it has to do.
    breakautoindent=true,    % Nach dem Zeilenumbruch Zeile einrücken.
    postbreak=\space,        % Bei Leerzeichen umbrechen.
    tabsize=2,               % Tabulatorsize 2
    basicstyle=\ttfamily\footnotesize, % Nichtproportionale Schrift, klein für den Quellcode
    showspaces=false,        % Leerzeichen nicht anzeigen.
    showstringspaces=false,  % Leerzeichen auch in Strings ('') nicht anzeigen.
    extendedchars=true,      % Alle Zeichen vom Latin1 Zeichensatz anzeigen.
    backgroundcolor=\color{black!5}, % Hintergrundfarbe des Quellcodes setzen.
    captionpos=b,         % Caption unter und nicht über das Listing schreiben
    frame = lines,
	floatplacement = tb,
	morekeywords = {foreach},
    %rulesepcolor=\color{darkgrey}
} 
%-----------------------------------------------------------------------------%
%                                                        Schrifteinstellungen %
%------------------------------------------------------------------------------
%\linespread{1.25}\selectfont%                                    mehr Abstand %
\linespread{1.10}\selectfont%                                    mehr Abstand %

%   Redefinition Typewriter-Befehls: kleinere Schrift, Silbentrennungszeichen %
\renewcommand\texttt[1]{% 
\small\ttfamily\hyphenchar\font=\defaulthyphenchar#1{}\normalsize\rmfamily\hyphenchar\font=\defaulthyphenchar}

\setkomafont{sectioning}{%        Abschnittsüberschriften serifenlos und fett %
\sffamily\bfseries}

\addtokomafont{caption}{%                            Bildunterschriften klein %
\normalfont\normalcolor\small}

\addtokomafont{captionlabel}{\small\bfseries}%             Bezeichnungen fett %


%------------------------------------------------------------------------------
%                             individuelle Formatierungen an TOC, LOF und LOT %
%------------------------------------------------------------------------------
\renewcommand{\cftchapfont}{%
\sffamily\bfseries}%                       Kapitelbezeichnung fett & serifenlos
\renewcommand{\cftchappagefont}{\sffamily
\bfseries}%                                     Kapitelnummer fett & serifenlos
\setlength{\cftbeforesecskip}{1.5pt}%           vert. Abstand vor Abschnittsbez.
\setlength{\cftbeforesubsecskip}{1pt}%     vert. Abstand vor Unterabschnittsbez.
\cftsetindents{chapter}{0em}{1.3em}%                         Einzug für Kapitel
\cftsetindents{section}{1.3em}{2em}%                      Einzug für Abschnitte
\cftsetindents{subsection}{3.3em}{2.7em}%            Einzug für Unterabschnitte
\cftsetindents{subsubsection}{6em}{3.3em}%           Einzug für Unterabschnitte


\setlength{\cftbeforetabskip}{1.5pt}%   vert. Abstand vor Tabellenbezeichnungen
\cftsetindents{table}{0em}{2.5em}%            Einzüge für Tabellenbezeichnungen
\renewcommand{\cfttabpresnum}{\hfill}%
\renewcommand{\cfttabaftersnum}{\hspace*{0.6em}}%

\setlength{\cftbeforefigskip}{1.5pt}%             Abstand vor Bildbezeichnungen
% \setlength{\cftbeforesubfigskip}{0.5pt}%     Abstand vor Unterbildbezeichnungen
% \cftsetindents{figure}{0em}{2.5em}%               Einzüge für Bildbezeichnungen
% \cftsetindents{subfigure}{2.5em}{1.8em}%     Einzüge für Unterbildbezeichnungen
% \renewcommand{\cftfigpresnum}{\hfill}%
\renewcommand{\cftfigaftersnum}{\hspace*{0.6em}}%

%-----------------------------------------------------------------------------%
%                                 Einstellung bestimmter Längen und Parameter %
%-----------------------------------------------------------------------------%
%\parindent0mm%                     KEIN Einzug bei neuem Absatz, statt dessen %
%\parskip2mm plus1.5pt minus1.5pt%              Abstand zwischen zwei Absätzen %
%\parindent5mm%                                        Einzug bei neuem Absatz %

%                 zusätzlicher vertikaler Abstand vor und nach Gleitobjekten, %
%                                    die mit der Option "h" eingefügt werden: %
\setlength{\intextsep}{1.5\baselineskip plus 2pt minus 2pt}%

%                           zusätzlicher vertikaler Abstand für Gleitobjekte, %
%                         die mit den Optionen "t" bzw. "b" eingefügt werden: %
\setlength{\textfloatsep}{1.5\baselineskip plus 2pt minus 2pt}

%\setcapwidth[c]{.8\textwidth}%    Abbildungsbeschriftungen 80% der Textbreite %

\addtolength{\footnotesep}{2pt}%      mehr Abstand zwischen Fußnoten und Text %

\pagestyle{scrheadings}%               Seitenstil mit lebenden Kolumnentiteln %
\clearpairofpagestyles
\ohead[]{\headmark}
\cfoot[]{}
\ofoot[\pagemark]{\pagemark}
% \setheadsepline{0.08em}%    Einstellungen der Dicke der Kopfzeilen-Trennlinie %

\setcounter{secnumdepth}{3}%                               Nummerierungstiefe %
\setcounter{tocdepth}{2}%                  Inhaltsverzeichnis bis zur Tiefe 2 %
\setcounter{lofdepth}{2}%               Abbildungsverzeichnis bis zur Tiefe 2 %

\renewcommand{\topfraction}{0.9}%      Anteile der Seite, die maximal/minimal %
\renewcommand{\bottomfraction}{0.9}%    von Abbildungen bzw. Text eingenommen %
\renewcommand{\textfraction}{0.1}%                              werden können %


%------------------------------------------------------------------------------
%                  Einstellungen für den Zeilenumbruchsalgorithmus, siehe dazu:
%           --> http://www.jr-x.de/publikationen/latex/tipps/zeilenumbruch.html
%------------------------------------------------------------------------------
\emergencystretch 1.5em%
\clubpenalty = 10000%
\widowpenalty = 10000%
\displaywidowpenalty = 10000
\doublehyphendemerits = 10000%
\tolerance=500

\hfuzz 0.25pt%                     Grenze, ab der "overfull hbox" gemeldet wird
\vfuzz \hfuzz%                     Grenze, ab der "overfull vbox" gemeldet wird

%------------------------------------------------------------------------------
%                              Einstellungen des Satzspiegels mit den Vorgaben:
%                  Rand innen: 35mm, Rand außen 25 mm, Rand oben/(unten) ~25 mm
%------------------------------------------------------------------------------

%%\paperwidth    597.50787pt%                            597,50787 pt = 210,00 mm
%%\paperheight   845.04684pt%                            845,04684 pt = 297,00 mm
%%\paperwidth    210mm
%%\paperheight   297mm
%%\textwidth     426.79135pt%       210 - 35 - 25 mm  =  426,79135 pt = 150,00 mm
%\textwidth 140mm
%%\textheight    638.40076pt%        41 Zeilen           621,07825 pt = 218,28 mm
%%\oddsidemargin  27.31465pt%                             27,31465 pt =   9,60 mm
%%\evensidemargin -1.13811pt%                              1,13811 pt =  -0,40 mm
%%\oddsidemargin  10mm
%%\evensidemargin 8.5mm
%\topmargin      -8.53583pt%                              8,53583 pt =  -3,00 mm
%\headheight     17.07165pt%                             17,07165 pt =   6,00 mm
%\headsep        19.91692pt%                             19,91692 pt =   7,00 mm
%\footskip       45.52440pt%                             45,52440 pt =  16,00 mm
%\marginparwidth 59.75079pt%                             59,75079 pt =  21,00 mm
%\marginparsep    8.53583pt%                              8,53583 pt =   3,00 mm
%\skip\footins  10.8pt plus 4.0pt minus 2.0pt
%%\hoffset 0.0pt%
%%\voffset 0.0pt%

\usepackage{todonotes}

% FOM-"Standard"
% \usepackage[style=verbose-trad2, backend=biber]{biblatex}
% \counterwithout*{footnote}{chapter}
% \AtEveryCitekey{%
% \clearfield{url}%
% \clearfield{isbn}%
% \clearfield{doi}%
% }
% Stern (IEEE)
\usepackage[style=ieee, mincitenames=1, maxcitenames=2, backend=biber]{biblatex}
\usepackage{xurl}%      URLs werden so auch im Literaturverzeichnis umgebrochen %

% \citetitle: Titel mit deutschen Anführungszeichen
\DeclareFieldFormat*{citetitle}{\glqq#1\grqq}

% "et al." für die Abkürzung mehrerer Autoren verwenden
\DefineBibliographyStrings{german}{%
  andothers = {\emph{et al}\adddot}
}

% Hier die entsprechende .bib-Datei eintragen
\addbibresource{Henning.bib}
\addbibresource{Alexander.bib}
\addbibresource{Paul.bib}


% Keine Warnung für geteilte Bibliographien
\BiblatexSplitbibDefernumbersWarningOff
\defbibheading{bibintoc}[\bibname]{%
  \addchap{#1}}
%%%%%%%%%%%%%%%%%%%%%%%%%%%%%%%%%%%%%%%%%%%%%%%%%%%%%%%%%%%%%%%%%%%%%%%%%%%%%%%
%                                                                             %
%               commands.tex - eigene Befehle für diverse Dinge               %
%                                                                             %
%%%%%%%%%%%%%%%%%%%%%%%%%%%%%%%%%%%%%%%%%%%%%%%%%%%%%%%%%%%%%%%%%%%%%%%%%%%%%%%

%-----------------------------------------------------------------------------%
%                                                   Grundlegende Definitionen %
%-----------------------------------------------------------------------------%
\author{Vorname Nachname}
\newcommand{\daAutor}{Paul Hornig, Henning Diederich und Alexander Lahn}
\newcommand{\daMatrikelnummer}{xxx, xxx, 732259}
\newcommand{\daTitel}{Ein besonderes Thema-das-interessant-ist}
\newcommand{\daTitelEins}{Verbesserung der RCA in Microservices:}
\newcommand{\daTitelZwei}{Integration eines Sprachmodells in Nezha zur}
\newcommand{\daTitelDrei}{präziseren Fehleridentifikation}
\newcommand{\daUnternehmen}{Unternehmen GmbH \& Co. KG}

\newcommand{\daUniURL}{http://www.fom.de}
\newcommand{\daUniLogo}[1]{\includegraphics[#1]{./bilder/FOM}}
\newcommand{\daStudiengang}{Bachelor Wirtschaftsinformatik}

\newcommand{\daAutorAdresse}{Musterstraße 123}
\newcommand{\daAutorPLZ}{12345}
\newcommand{\daAutorOrt}{Musterstadt}

\newcommand{\daDate}{1.2.2021}


\newcommand{\daGutachterEins}{Prof. Dr. Sascha Schlauberger}

%-----------------------------------------------------------------------------%
%                                                   einige Mathe-Definitionen %
%-----------------------------------------------------------------------------%
\newcommand{\CoG}{\operatorname{CoG}}

%-----------------------------------------------------------------------------%
%                                                  Befehl für das BibTeX-Logo %
%-----------------------------------------------------------------------------%
\makeatletter
\DeclareRobustCommand{\BibTeX}{B\hbox{\check@mathfonts
\fontsize\sf@size\z@
\math@fontsfalse\selectfont
IB}\kern-.08em
\TeX}
\makeatother

%-----------------------------------------------------------------------------%
%      Punktereihe mit größerem Abstand zwischen den Punkten als bei \dotfill %
%-----------------------------------------------------------------------------%
\makeatletter
\def\punktfill{\cleaders\hbox{$\m@th \mkern4mu . \mkern4mu$}\hfill}
\makeatother

%-----------------------------------------------------------------------------%
%                                    Makros zur Festlegung von PDF-Attributen %
%-----------------------------------------------------------------------------%
\newcommand{\setPDFCreationDate}{%               CreationDate ist Abgabedatum %
\pdfinfo{%
/CreationDate (D:%
\thedateyear%
\ifnum\value{datemonth}<10 0\fi%
\thedatemonth%
\ifnum\value{dateday}<10 0\fi%
\thedateday%
010000+01'00')}%
}
%-----------------------------------------------------------------------------%
\newcommand{\setPDFModDateToCreationDate}{% ModDate ist ebenfalls Abgabedatum %
\pdfinfo{/CreationDate (D:\thedateyear%
\ifnum\value{datemonth}<10 0\fi\thedatemonth%
\ifnum\value{dateday}<10 0\fi\thedateday010000+01'00')}}
%-----------------------------------------------------------------------------%


\gdef\twodigits#1{\ifnum#1<10 0\fi\the#1}
%-----------------------------------------------------------------------------%
\newcommand{\setPDFModDateToCurrentDate}{%        ModDate ist aktuelles Datum
\begingroup
\count0=\time \divide\count0 by 60
\edef\x{\twodigits{\count0}}%
\multiply\count0 by 60
\count1=\time \advance\count1 by -\count0
\edef\x{\x\twodigits{\count1}}%
\edef\x{/ModDate (D:\the\year 
\twodigits\month \twodigits\day \x 00+01'00')}%
\expandafter\endgroup
\expandafter\pdfinfo\expandafter{\x}%
}

%-----------------------------------------------------------------------------%
%                  Titel, Thema, Author, Schlüsselwörter und Erzeugerprogramm %
%-----------------------------------------------------------------------------%
\newcommand{\daHyperSetup}{%
\hypersetup{%
pdftitle    = {\protect{\daTitel}},%
pdfsubject  = {Diplomarbeit an der Universität Paderborn},%
pdfauthor   = {\protect{\daAutor}},%
pdfkeywords = {\daPDFKeywords},%
pdfcreator = {LaTeX2e mit {\frqq KOMA-Script\flqq} Book-Klasse und {\frqq hyperref\flqq}-Package}}
}
%-----------------------------------------------------------------------------%
%          Befehle für bestimmte deutsche Abkürzungen mit passenden Abständen %
%                                         (kein Anspruch auf Vollständigkeit) %
%-----------------------------------------------------------------------------%
\newlength{\abkz}
\setlength{\abkz}{1pt}
\newcommand{\abkrz}{\hspace{\abkz}}

\newcommand{\bzw}{bzw.\@\xspace}
\newcommand{\bzgl}{bezüglich\@\xspace}
\newcommand{\ca}{ca.\@\xspace}
\newcommand{\dah}{\mbox{d.{\abkrz}h.}\@\xspace}
\newcommand{\Dah}{D.{\abkrz}h.\@\xspace}
\newcommand{\ds}{d.{\abkrz}s.\@\xspace}
\newcommand{\evtl}{evtl.\@\xspace}
\newcommand{\etc}{etc.\@\xspace}
\newcommand{\ua}{u.{\abkrz}a.\@\xspace}
\newcommand{\Ua}{U.{\abkrz}a.\@\xspace}
\newcommand{\usw}{usw.\@\xspace}
\newcommand{\va}{v.{\abkrz}a.\@\xspace}
\newcommand{\vgl}{vgl.\@\xspace}
\newcommand{\zB}{\mbox{z.{\abkrz}B.}\@\xspace}
\newcommand{\ZB}{Zum Beispiel\xspace}
\newcommand{\zT}{z.{\abkrz}T.\@\xspace}
\newcommand{\ZT}{Zum Teil\xspace}
\newcommand{\zZ}{zur Zeit\xspace}
\newcommand{\sa}{s.{\abkrz}a.\@\xspace}
\newcommand{\ia}{i.{\abkrz}a.\@\xspace}
\newcommand{\uU}{u.{\abkrz}U.\@\xspace}
\newcommand{\bspw}{beispielsweise\@\xspace}
\newcommand{\Bspw}{Beispielsweise\@\xspace}
\newcommand{\ggf}{gegebenenfalls\@\xspace}
\newcommand{\sog}{sogenannte\@\xspace}
\newcommand{\soger}{sogenannter\@\xspace}
\newcommand{\sogen}{sogenannten\@\xspace}
\newcommand{\soges}{sogenanntes\@\xspace}
\newcommand{\og}{o.\,g.\@\xspace}

%-----------------------------------------------------------------------------%
%        erzeugt eingeklammerten, kursiven Begriff mit dem Wort "engl." davor %
%-----------------------------------------------------------------------------%
\newcommand{\engl}[1]{%
(engl. \emph{#1})\@\xspace}%

%-----------------------------------------------------------------------------%
%                    Befehle für das Verzeichnis der Akronyme und Abkürzungen %
%-----------------------------------------------------------------------------%
\newlength{\acronWidth}
\newcommand{\acron}[2]{%
\protect\hyperdef{acDef}{#1}{\rule{0pt}{1pt}}\\[-3ex]%
\makebox[\acronWidth][l]{#1}~~{#2}\par%
}
\newcommand{\acronShort}[1]{\protect\hyperlink{acDef.#1}{#1}}
\newcommand{\acronShortPlural}[1]{\protect\hyperlink{acDef.#1}{#1{s}}}


%-----------------------------------------------------------------------------%
%                         Neudefinition der Referenzierungs-Befehle (kleiner) %
%-----------------------------------------------------------------------------%
%\makeatletter\renewcommand{\@cite}[2]{[{\small #1\if@tempswa , #2\fi}]}\makeatother
%\makeatletter\renewcommand{\@biblabel}[1]{[{\small #1}]}\makeatother

%-----------------------------------------------------------------------------%
%   Gestaltung des Datei-Verweises zur CD-ROM, falls BibTeX-Item cdlink = {1} %
%                                         (siehe dazu die BibTeX-Style-Datei) %
%-----------------------------------------------------------------------------%
\newcommand{\CDLink}[1]{%
~\mbox{$\rightarrow$ {\itshape\footnotesize %
Auf beiliegender CD:}~~\href{file:./literatur/#1}{%
\raisebox{-0.5pt}{\includegraphics[height=7.5pt]{cdIconBlue}}%
\textcolor{diplaBlue}{\ttfamily\footnotesize/literatur/#1.pdf}}}}

%-----------------------------------------------------------------------------%
\newcommand{\blindtext}{%                                eigener Blindtext ;) %
%-----------------------------------------------------------------------------%
Weit hinten, hinter den Wortbergen, fern der Länder Vokalien und 
Konsonantien leben die Blindtexte. Abgeschieden wohnen sie in 
Buchstabhausen an der Küste des Semantik, eines großen Sprachozeans. 
Ein kleines Bächlein namens Duden fließt durch ihren Ort und versorgt 
sie mit den nötigen Regelialien. Es ist ein paradiesmatisches Land, 
in dem einem gebratene Satzteile in den Mund fliegen. Nicht einmal 
von der allmächtigen Interpunktion werden die Blindtexte 
beherrscht -- ein geradezu unorthographisches Leben. 

Eines Tages aber beschloß eine kleine Zeile Blindtext, 
ihr Name war Lorem Ipsum, hinaus zu gehen in die weite Grammatik. 
Der große Oxmox riet ihr davon ab, da es dort wimmele von bösen Kommata, 
wilden Fragezeichen und hinterhältigen Semikola, doch das Blindtextchen 
ließ sich nicht beirren. Es packte seine sieben Versalien, schob sich 
sein Initial in den Gürtel und machte sich auf den Weg. 
Als es die ersten Hügel des Kursivgebirges erklommen hatte, warf es 
einen letzten Blick zurück auf die Skyline seiner Heimatstadt Buchstabhausen, 
die Headline von Alphabetdorf und die Subline seiner eigenen Straße, 
der Zeilengasse. Wehmütig lief ihm eine rhetorische Frage über die Wange, 
dann setzte es seinen Weg fort. Unterwegs traf es eine Copy. Die Copy 
warnte das Blindtextchen, da, wo sie her käme wäre sie zigmal umgeschrieben 
worden und alles, was von ihrem Ursprung noch übrig wäre, sei das Wort "`und"' 
und das Blindtextchen solle umkehren und wieder in sein eigenes, sicheres Land 
zurückkehren. Doch alles Gut zureden konnte es nicht überzeugen und so dauerte 
es nicht lange, bis ihm ein paar heimtückische Werbetexter auflauerten, es mit 
Langue und Parole betrunken machten und es dann in ihre Agentur schleppten, 
wo sie es für ihre Projekte wieder und wieder mißbrauchten. Und wenn es nicht 
umgeschrieben wurde, dann benutzen sie es immer noch.%
}



%=============================================================================%
%                                 Befehle zum Setzen eines Verzeichnis-Baumes %
%=============================================================================%
\newcounter{dirCounter}
\newcommand{\cdIcon}[1]{%
\raisebox{-2pt}{\includegraphics[width=10pt]{cdIcon}}~%
{\ttfamily\bfseries\itshape\small #1}\@\xspace
}
%-----------------------------------------------------------------------------%
%                                     Befehle für Ordner/Dateien auf 1. Ebene %
%-----------------------------------------------------------------------------%
%                                offener Order mit mit Folgemuster:  E        %
%-----------------------------------------------------------------------------%
\newcommand{\oEm}[2][21]{%
\begin{picture}(12,6)
  \linethickness{.6pt}
  \put(5,-6){\line(0,1){#1}}
  \put(5,3){\line(1,0){5}}
  \put(5,3){\makebox(0,0){\includegraphics[width=5pt]{minus}}}
\end{picture}%
\raisebox{-0.5pt}{\includegraphics[width=12pt]{oOrdner}}~%
{\ttfamily\bfseries\itshape\footnotesize #2}\@\xspace
}
%-----------------------------------------------------------------------------%
%                          geschlossener Order mit mit Folgemuster:  E        %
%-----------------------------------------------------------------------------%
\newcommand{\oEp}[2][21]{%
\begin{picture}(12,6)
  \linethickness{.6pt}
  \put(5,-5){\line(0,1){#1}}
  \put(5,3){\line(1,0){5}}
  \put(5,3){\makebox(0,0){\includegraphics[width=5pt]{plus}}}
\end{picture}%
\raisebox{-0.5pt}{\includegraphics[width=12pt]{ordner}}~%
{\ttfamily\bfseries\itshape\footnotesize #2}\@\xspace
}
%-----------------------------------------------------------------------------%
%                                    offener Order mit Folgemuster:  L        %
%-----------------------------------------------------------------------------%
\newcommand{\oLm}[2][13]{%
\begin{picture}(12,6)
  \linethickness{.6pt}
  \put(5,3){\line(0,1){#1}}
  \put(5,3){\line(1,0){5}}
  \put(5,3){\makebox(0,0){\includegraphics[width=5pt]{minus}}}
\end{picture}%
\raisebox{-0.5pt}{\includegraphics[width=12pt]{oOrdner}}~%
{\ttfamily\bfseries\itshape\footnotesize #2}\@\xspace
}
%-----------------------------------------------------------------------------%
%                              geschlossener Order mit Folgemuster:  L        %
%-----------------------------------------------------------------------------%
\newcommand{\oLp}[2][13]{%
\begin{picture}(12,6)
  \linethickness{.6pt}
  \put(5,3){\line(0,1){#1}}
  \put(5,3){\line(1,0){5}}
  \put(5,3){\makebox(0,0){\includegraphics[width=5pt]{plus}}}
\end{picture}%
\raisebox{-0.5pt}{\includegraphics[width=12pt]{ordner}}~%
{\ttfamily\bfseries\itshape\footnotesize #2}\@\xspace
}
%-----------------------------------------------------------------------------%
%                                                          Datei auf 1. Ebene %
%-----------------------------------------------------------------------------%
\newcommand{\dI}[3][21]{%
\begin{picture}(12,6)
  \linethickness{.6pt}
  \put(5,-5){\line(0,1){#1}}
\end{picture}{\ttfamily\footnotesize #2}~\punktfill~{\small #3}}
%-----------------------------------------------------------------------------%

%-----------------------------------------------------------------------------%
%                                 Befehle für Ordner/Dateien auf der 2. Ebene %
%-----------------------------------------------------------------------------%
%                                    offener Order mit Folgemuster:  I  E     %
%-----------------------------------------------------------------------------%
\newcommand{\oIEm}[2][21]{%
\begin{picture}(24,6)
  \linethickness{.6pt}
  \put(5,-5){\line(0,1){#1}}
  \put(17,-5){\line(0,1){#1}}
  \put(17,3){\line(1,0){5}}
  \put(17,3){\makebox(0,0){\includegraphics[width=5pt]{minus}}}
  \end{picture}%
\raisebox{-0.5pt}{\includegraphics[width=12pt]{oOrdner}}~%
{\ttfamily\bfseries\itshape\footnotesize #2}\@\xspace
}
%-----------------------------------------------------------------------------%
%                              geschlossener Order mit Folgemuster:  I  E     %
%-----------------------------------------------------------------------------%
\newcommand{\oIEp}[2][21]{%
\begin{picture}(24,6)
  \linethickness{.6pt}
  \put(5,-5){\line(0,1){#1}}
  \put(17,-5){\line(0,1){#1}}
  \put(17,3){\line(1,0){5}}
  \put(17,3){\makebox(0,0){\includegraphics[width=5pt]{plus}}}
  \end{picture}%
\raisebox{-0.5pt}{\includegraphics[width=12pt]{ordner}}~%
{\ttfamily\bfseries\itshape\footnotesize #2}\@\xspace
}
%-----------------------------------------------------------------------------%
%                                    offener Order mit Folgemuster:  I  L     %
%-----------------------------------------------------------------------------%
\newcommand{\oILm}[2][21]{%
\setcounter{dirCounter}{#1}
\addtocounter{dirCounter}{-8}
\begin{picture}(24,6)
  \linethickness{.6pt}
  \put(5,-5){\line(0,1){#1}}
  \put(17,3){\line(0,1){\value{dirCounter}}}
  \put(17,3){\line(1,0){5}}
  \put(17,3){\makebox(0,0){\includegraphics[width=5pt]{minus}}}
\end{picture}%
\raisebox{-0.5pt}{\includegraphics[width=12pt]{oOrdner}}~%
{\ttfamily\bfseries\itshape\footnotesize #2}\@\xspace
}
%-----------------------------------------------------------------------------%
%                              geschlossener Order mit Folgemuster:  I  L     %
%-----------------------------------------------------------------------------%
\newcommand{\oILp}[2][21]{%
\setcounter{dirCounter}{#1}
\addtocounter{dirCounter}{-8}
\begin{picture}(24,6)
  \linethickness{.6pt}
  \put(5,-5){\line(0,1){#1}}
  \put(17,3){\line(0,1){\value{dirCounter}}}
  \put(17,3){\line(1,0){5}}
  \put(17,3){\makebox(0,0){\includegraphics[width=5pt]{plus}}}
\end{picture}%
\raisebox{-0.5pt}{\includegraphics[width=12pt]{ordner}}~%
{\ttfamily\bfseries\itshape\footnotesize #2}\@\xspace
}
%-----------------------------------------------------------------------------%
%                                    offener Order mit Folgemuster:     E     %
%-----------------------------------------------------------------------------%
\newcommand{\oXEm}[2][21]{%
\begin{picture}(24,6)
  \linethickness{.6pt}
  \put(17,-5){\line(0,1){#1}}
  \put(17,3){\line(1,0){5}}
  \put(17,3){\makebox(0,0){\includegraphics[width=5pt]{minus}}}
  \end{picture}%
\raisebox{-0.5pt}{\includegraphics[width=12pt]{oOrdner}}~%
{\ttfamily\bfseries\itshape\footnotesize #2}\@\xspace
}
%-----------------------------------------------------------------------------%
%                              geschlossener Order mit Folgemuster:     E     %
%-----------------------------------------------------------------------------%
\newcommand{\oXEp}[2][21]{%
\begin{picture}(24,6)
  \linethickness{.6pt}
  \put(17,-5){\line(0,1){#1}}
  \put(17,3){\line(1,0){5}}
  \put(17,3){\makebox(0,0){\includegraphics[width=5pt]{plus}}}
  \end{picture}%
\raisebox{-0.5pt}{\includegraphics[width=12pt]{ordner}}~%
{\ttfamily\bfseries\itshape\footnotesize #2}\@\xspace
}
%-----------------------------------------------------------------------------%
%                                    offener Order mit Folgemuster:     L     %
%-----------------------------------------------------------------------------%
\newcommand{\oXLm}[2][13]{%
\begin{picture}(24,6)
  \linethickness{.6pt}
  \put(17,3){\line(0,1){#1}}
  \put(17,3){\line(1,0){5}}
  \put(17,3){\makebox(0,0){\includegraphics[width=5pt]{minus}}}
\end{picture}%
\raisebox{-0.5pt}{\includegraphics[width=12pt]{oOrdner}}~%
{\ttfamily\bfseries\itshape\footnotesize #2}\@\xspace
}
%-----------------------------------------------------------------------------%
%                              geschlossener Order mit Folgemuster:     L     %
%-----------------------------------------------------------------------------%
\newcommand{\oXLp}[2][13]{%
\begin{picture}(24,6)
  \linethickness{.6pt}
  \put(17,3){\line(0,1){#1}}
  \put(17,3){\line(1,0){5}}
  \put(17,3){\makebox(0,0){\includegraphics[width=5pt]{plus}}}
\end{picture}%
\raisebox{-0.5pt}{\includegraphics[width=12pt]{ordner}}~%
{\ttfamily\bfseries\itshape\footnotesize #2}\@\xspace
}
%-----------------------------------------------------------------------------%
%                                            Datei mit Folgemuster:  I        %
%-----------------------------------------------------------------------------%
\newcommand{\dII}[3][23]{%
\begin{picture}(23,6)
  \linethickness{.6pt}
  \put(5,-5){\line(0,1){#1}}
  \put(17,-5){\line(0,1){#1}}
\end{picture}~%
\texttt{\footnotesize #2}~\punktfill~{\small #3}}
%-----------------------------------------------------------------------------%
%                                            Datei mit Folgemuster:           %
%-----------------------------------------------------------------------------%
\newcommand{\dXI}[3][23]{%
\begin{picture}(23,6)
  \linethickness{.6pt}
  \put(17,-5){\line(0,1){#1}}
\end{picture}~%
\texttt{\footnotesize #2}~\punktfill~{\small #3}}
%-----------------------------------------------------------------------------%
%                                 Befehle für Ordner/Dateien auf der 3. Ebene %
%-----------------------------------------------------------------------------%
%                                    offener Order mit Folgemuster:  I  I  L  %
%-----------------------------------------------------------------------------%
\newcommand{\oIILm}[2][21]{%
\setcounter{dirCounter}{#1}
\addtocounter{dirCounter}{-8}
\begin{picture}(36,6)
  \linethickness{.6pt}
  \put(5,-5){\line(0,1){#1}}
  \put(17,-5){\line(0,1){#1}}
  \put(29,3){\line(0,1){\value{dirCounter}}}
  \put(29,3){\line(1,0){5}}
  \put(29,3){\makebox(0,0){\includegraphics[width=5pt]{minus}}}
\end{picture}%
\raisebox{-0.5pt}{\includegraphics[width=12pt]{oOrdner}}~%
{\ttfamily\bfseries\itshape\footnotesize #2}\@\xspace
}
%-----------------------------------------------------------------------------%
%                              geschlossener Order mit Folgemuster:  I  I  L  %
%-----------------------------------------------------------------------------%
\newcommand{\oIILp}[2][21]{%
\setcounter{dirCounter}{#1}
\addtocounter{dirCounter}{-8}
\begin{picture}(36,6)
  \linethickness{.6pt}
  \put(5,-5){\line(0,1){#1}}
  \put(17,-5){\line(0,1){#1}}
  \put(29,3){\line(0,1){\value{dirCounter}}}
  \put(29,3){\line(1,0){5}}
  \put(29,3){\makebox(0,0){\includegraphics[width=5pt]{plus}}}
\end{picture}%
\raisebox{-0.5pt}{\includegraphics[width=12pt]{ordner}}~%
{\ttfamily\bfseries\itshape\footnotesize #2}\@\xspace
}
%-----------------------------------------------------------------------------%
%                                    offener Order mit Folgemuster:  I  I  E  %
%-----------------------------------------------------------------------------%
\newcommand{\oIIE}[2][21]{%
\begin{picture}(36,6)
  \linethickness{.6pt}
  \put(5,-5){\line(0,1){#1}}
  \put(17,-5){\line(0,1){#1}}
  \put(29,-5){\line(0,1){#1}}
  \put(29,3){\line(1,0){5}}
  \put(29,3){\makebox(0,0){\includegraphics[width=5pt]{minus}}}
\end{picture}%
\raisebox{-0.5pt}{\includegraphics[width=10pt]{ordner}}~%
{\ttfamily\bfseries\itshape\footnotesize #2}\@\xspace
}
%-----------------------------------------------------------------------------%
%                              geschlossener Order mit Folgemuster:  I  I  E  %
%-----------------------------------------------------------------------------%
\newcommand{\oIIEp}[2][21]{%
\begin{picture}(36,6)
  \linethickness{.6pt}
  \put(5,-5){\line(0,1){#1}}
  \put(17,-5){\line(0,1){#1}}
  \put(29,-5){\line(0,1){#1}}
  \put(29,3){\line(1,0){5}}
  \put(29,3){\makebox(0,0){\includegraphics[width=5pt]{plus}}}
\end{picture}%
\raisebox{-0.5pt}{\includegraphics[width=12pt]{ordner}}~%
{\ttfamily\bfseries\itshape\footnotesize #2}\@\xspace
}
%-----------------------------------------------------------------------------%
%                              geschlossener Order mit Folgemuster:  I     L  %
%-----------------------------------------------------------------------------%
\newcommand{\oIXLp}[2][21]{%
\setcounter{dirCounter}{#1}
\addtocounter{dirCounter}{-8}
\begin{picture}(36,6)
  \linethickness{.6pt}
  \put(5,-5){\line(0,1){#1}}
  \put(29,3){\line(0,1){\value{dirCounter}}}
  \put(29,3){\line(1,0){5}}
  \put(29,3){\makebox(0,0){\includegraphics[width=5pt]{plus}}}
\end{picture}%
\raisebox{-0.5pt}{\includegraphics[width=10pt]{ordner}}~%
{\ttfamily\bfseries\itshape\footnotesize #2}\@\xspace
}
%-----------------------------------------------------------------------------%
%                                    offener Order mit Folgemuster:  I     L  %
%-----------------------------------------------------------------------------%
\newcommand{\oIXLm}[2][21]{%
\setcounter{dirCounter}{#1}
\addtocounter{dirCounter}{-8}
\begin{picture}(36,6)
  \linethickness{.6pt}
  \put(5,-5){\line(0,1){#1}}
  \put(29,3){\line(0,1){\value{dirCounter}}}
  \put(29,3){\line(1,0){5}}
  \put(29,3){\makebox(0,0){\includegraphics[width=5pt]{minus}}}
\end{picture}%
\raisebox{-0.5pt}{\includegraphics[width=10pt]{oOrdner}}~%
{\ttfamily\bfseries\itshape\footnotesize #2}\@\xspace
}
%-----------------------------------------------------------------------------%
%                              geschlossener Order mit Folgemuster:  I     E  %
%-----------------------------------------------------------------------------%
\newcommand{\oIXEp}[2][21]{%
\begin{picture}(36,6)
  \linethickness{.6pt}
  \put(5,-5){\line(0,1){#1}}
  \put(29,-5){\line(0,1){#1}}
  \put(29,3){\line(1,0){5}}
  \put(29,3){\makebox(0,0){\includegraphics[width=5pt]{plus}}}
\end{picture}%
\raisebox{-0.5pt}{\includegraphics[width=12pt]{ordner}}~%
{\ttfamily\bfseries\itshape\footnotesize #2}\@\xspace
}
%-----------------------------------------------------------------------------%
%                                    offener Order mit Folgemuster:  I     E  %
%-----------------------------------------------------------------------------%
\newcommand{\oIXE}[2][21]{%
\begin{picture}(36,6)
  \linethickness{.6pt}
  \put(5,-5){\line(0,1){#1}}
  \put(29,-5){\line(0,1){#1}}
  \put(29,3){\line(1,0){5}}
  \put(29,3){\makebox(0,0){\includegraphics[width=5pt]{minus}}}
\end{picture}%
\raisebox{-0.5pt}{\includegraphics[width=10pt]{ordner}}~%
{\ttfamily\bfseries\itshape\footnotesize #2}\@\xspace
}
%------------------------------------------------------------------------------
\newcommand{\dIII}[3][21]{%
\begin{picture}(36,6)
  \linethickness{.6pt}
  \put(5,-5){\line(0,1){#1}}
  \put(17,-5){\line(0,1){#1}}
  \put(29,-5){\line(0,1){#1}}
\end{picture}{\ttfamily\footnotesize #2}~\punktfill~{\small #3}}
%------------------------------------------------------------------------------
\newcommand{\dIIx}[3][21]{%
\begin{picture}(36,6)
  \linethickness{.6pt}
  \put(5,-5){\line(0,1){#1}}
  \put(17,-5){\line(0,1){#1}}
\end{picture}{\ttfamily\footnotesize #2}~\punktfill~{\small #3}}
%------------------------------------------------------------------------------




\AtBeginDocument{%
  \renewcommand*{\subsectionautorefname}{\sectionautorefname}%
  \renewcommand*{\subsubsectionautorefname}{\subsectionautorefname}%
  \renewcommand*{\paragraphautorefname}{\subsubsectionautorefname}%
  \renewcommand*{\subparagraphautorefname}{\paragraphautorefname}%
  \renewcommand*{\lstlistlistingname}{Quellcodeverzeichnis}%
}
% \newcommand{\todo}[1]{\marginpar{\color{red}#1}}
\newcommand{\todotext}[1]{\todo[inline]{#1}}
\newcommand{\note}[1]{\todo{{#1}}}

\hyphenation{}
\begin{document}

\frontmatter

\newlength{\logoWidth}
\newlength{\spalteEins}
\newlength{\spalteZwei}

\settowidth{\logoWidth}{\fbox{\normalsize\daStudiengang}}
\addtolength{\logoWidth}{-1ex}

%-----------------------------------------------------------------------------%
\begin{titlepage}%                                Formatierung der Titelseite %
%-----------------------------------------------------------------------------%
\begin{center}
  \sffamily
  \pdfbookmark{Titel}{title}%
  \href{\daUniURL}{\includegraphics[width=\logoWidth]{bilder/FOM.jpg}} \\ %
%  \vspace{\stretch{1}}
%  \href{\daFakultaetURL}{\normalsize \daFakultaet} \\%
%  \href{\daInstitutURL}{\normalsize \daInstitut} \\%
%  \href{\daFachgebietURL}{\normalsize \daFachgebiet} \\%
%  {\normalsize \daUniAdresse} \\%
%  {\normalsize \daUniPLZ~\daUniOrt}\\%
  \vspace{\stretch{12}}%
  \hrule
  \begin{spacing}{1.8}%
	\vspace{\stretch{5}} %
    {\sffamily\bfseries\LARGE\daTitelEins}\\ %
    {\sffamily\bfseries\LARGE\daTitelZwei}\\ %
    {\sffamily\bfseries\LARGE\daTitelDrei}\\ %
	\vspace{\stretch{4}} %
	{\sffamily\LARGE\daAutor}\\ %	    
	\vspace{\stretch{3}} %
	{\sffamily\large Matrikelnummer: \daMatrikelnummer}\\ %	    
	\vspace{\stretch{3}} %
  \end{spacing}%
  \hrule
  \vspace{\stretch{10}}%
  \begin{spacing}{1.1}
    {\sffamily\bfseries\Large Abschlussarbeit}\\ % Zu ersetzen!
    \vspace{\stretch{3}}%
    {\large zur Erlangung des akademischen Grades}\\
    \vspace{\stretch{3}}%
    {\large\bfseries Bachelor of Science (B.Sc.)}\\
%    \vspace{\stretch{3}}%
%    {\large der Fakultät}\\
    \vspace{\stretch{3}}%
  	{\large im Studiengang {\large \daStudiengang}} \\%
    \vspace{\stretch{3}}%
  	{\large der \href{\daUniURL}{FOM Hochschule für Oekonomie \& Management}} \\%
    \vspace{\stretch{8}}%
    {\normalsize\daAutor}\\%
    {\normalsize\daAutorAdresse}\\%
    {\normalsize\daAutorPLZ~\daAutorOrt}\\%
    \vspace{\stretch{8}}%
    {\small vorgelegt bei}\\%
    \vspace{\stretch{2}}%
    {\normalsize\daGutachterEins}\par%
%    \vspace{\stretch{2}}%
%    {\small und}\\%
%    \vspace{\stretch{2}}%
%    {\normalsize\daGutachterZwei}\par
  \end{spacing}
% \enlargethispage{1em}
  \vspace{\stretch{8}}%
  {\normalsize \daAutorOrt, \daDate}
\end{center}

\settowidth{\spalteEins}{\fbox{Mündliche Prüfung:}}
\setlength{\spalteZwei}{\textwidth}
\addtolength{\spalteZwei}{-\spalteEins}

%-----------------------------------------------------------------------------%
\clearpage%                                   Formatierung der Titelrückseite %
%-----------------------------------------------------------------------------%
\thispagestyle{empty}%
\noindent%
%\pdfbookmark[1]{Autor, Gutachter und Betreuer}{aut}%
%\href{\daUniURL}{\daUniLogo{width=\logoWidth}}\\%
%\href{\daFakultaetURL}{\daFakultaet} \\%
%\href{\daInstitutURL}{\daInstitut} \\%
%\href{\daFachgebietURL}{\daFachgebiet} \\%
%\daUniAdresse \\%
%\daUniPLZ~\daUniOrt \\%
%\vfill %
%\begin{tabular}{@{}p{\spalteEins}@{}p{\spalteZwei}@{}}
%  Verfasser: & \daAutor \\
%             & \daAutorAdresse \\
%             & \daAutorPLZ~\daAutorOrt \\[2ex]
%  Gutachter: & \daGutachterEins \\[0.5ex]
%             & \daGutachterZwei \\[2ex]
%  Mündliche Prüfung: & TBD \\[2ex]
%  Stand vom: & \thedateday.~\datemonthname~\thedateyear\\[6ex]
%\end{tabular}%
%\par%
%\vfill%
%\enlargethispage{2em}%

%\copyright~\thedateyear~\daAutor
%\daRevision

%-----------------------------------------------------------------------------%
\end{titlepage}
%-----------------------------------------------------------------------------%

%\thispagestyle{\chapterpagestyle}
%\vspace*{2.4\baselineskip}
%%-----------------------------------------------------------------------------%
%{\sffamily\bfseries\LARGE Erklärung}\par
%%-----------------------------------------------------------------------------%
%\vspace{1.725\baselineskip plus .115\baselineskip minus .192\baselineskip}
%Hiermit versichere ich, dass ich die vorliegende wissenschaftliche Arbeit selbständig und ohne Hilfe Dritter verfasst, keine anderen als die angegebenen Quellen und Hilfsmittel verwendet sowie alle wörtlichen oder sinngemäßen Entlehnungen unter Angabe der Quelle deutlich als solche gekennzeichnet habe.
%
%Diese Arbeit hat in gleicher oder ähnlicher Form meines Wissens noch keiner Prüfungsbehörde vorgelegen oder ist von dieser als Teil einer Prüfungsleistung angenommen worden.
%
%\vspace*{12ex}
%
%\daAutorOrt, \datedate


\cleardoublepage
\chapter*{Zusammenfassung}

Hier steht eine kurze Inhaltszusammenfassung von etwa einer Seite.
Es soll der komplette Inhalt zusammengefasst werden, also insbesondere können auch bereits Ergebnisse genannt werden.

% \cleardoublepage
% \chapter*{Abstract}
% At this position, a short abstract is to be written. An abstract covers the complete work, especially also results. A reader should get a good overview over the results of this work and the methods to achieve them. %das ist/kann ein Abstract sein

\chapter{Vorwort}

Das in dieser Arbeit gewählte generische Maskulinum bezieht sich zugleich auf die männliche, die weibliche und andere Geschlechteridentitäten. Zur besseren Lesbarkeit wird auf die gleichzeitige Verwendung männlicher und weiblicher Sprachformen verzichtet. Alle Geschlechteridentitäten werden ausdrücklich mitgemeint, soweit die Aussagen dies erfordern.


% Bitte Kommentarzeichen entfernen, falls KI als "mentaler Sparringspartner" verwendet wurde.
% Die Copy&Paste-Übernahme fällt hier nicht drunter! Dazu bitte den Leitfaden konsultieren.
%
%In der vorliegenden Arbeit wird der Einsatz von künstlicher Intelligenz (KI), beispielsweise des Sprachverarbeitungsmodells ChatGPT, als Instrument zur Unterstützung bei Textformulierungen genutzt. Die tranformerbasierten Sprachmodelle werden als interaktives Tool zur Überprüfung und Verbesserung von wissenschaftlichen Texten eingesetzt, indem Vorschläge zur Sprachverfeinerung und zur logischen Strukturierung von Argumenten geboten werden.
%
%Es ist wichtig zu betonen, dass die Verwendung von KI in dieser Arbeit transparent und ethisch verantwortungsvoll erfolgt. Alle durch die KI generierten Inhalte werden sorgfältig überprüft, um sicherzustellen, dass sie den wissenschaftlichen Standards entsprechen und einen Beitrag zur Forschung leisten. Die finale Verantwortung für den Inhalt der Arbeit liegt beim Autor.

\cleardoublepage
\pdfbookmark{\contentsname}{Inhaltsverzeichnis}
\tableofcontents

%-----------------------------------------------------------------------------%
% Verzeichnisse. Bitte leere Verzeichnisse auskommentieren.                   %
%-----------------------------------------------------------------------------%
\listoffigures
\listoftables
%\lstlistoflistings
\chapter*{Abkürzungsverzeichnis}\markboth{}{Abkürzungsverzeichnis}
\addcontentsline{toc}{chapter}{Abkürzungsverzeichnis}
\begin{acronym}[JSON] % Dort, wo "JSON" steht, kommt die längste Abkürzung hin, damit wird die Ausrichtung definiert.
% Bitte die Abkürzungen alphabetisch sortieren. Das muss manuell geschehen.
% Achtung! Nur wirklich verwendete Abkürzungen erscheinen im Abkürzungsverzeichnis.
 \acro{AP}{access point}
 \acro{API}{application programming interface}
 \acro{CRISP-DM}Cross-industry standard process for data mining
 \acro{JSON}{JavaScript Object Notation}
 \acro{REST}{Representational State Transfer}
 \acro{LLM}{Large Language Model}
 \acro{API}{Application Programming Interface}
\end{acronym}
%-----------------------------------------------------------------------------%
% Sperrvermerk. Bitte möglichst nicht verwenden.                              %
%-----------------------------------------------------------------------------%
%\chapter*{Sperrvermerk}\markboth{}{Sperrvermerk}
\addcontentsline{toc}{chapter}{Sperrvermerk}
Die vorliegende Abschlussarbeit mit dem Titel \daTitel{} enthält unternehmensinterne Daten der Firma \daUnternehmen{}. Daher ist sie nur zur Vorlage bei der FOM sowie den Begutachtern der Arbeit bestimmt. Für die Öffentlichkeit und dritte Personen darf sie nicht zugänglich sein.

\section*{Lock Flag}
The present thesis with the title \daTitel{} contains in-house data of the company \daUnternehmen{}. Herefore it is only assigned for the FOM as well as the assessors of the thesis. The thesis must not be made publicly available, neither made available to third persons.
\vspace{2cm}

\noindent
(Ort, Datum)\hfill						(Eigenhändige Unterschrift)

\noindent
(Place, Date)\hfill						(Personal Signature)



%-----------------------------------------------------------------------------%
% Hauptteil                                                                   %
%-----------------------------------------------------------------------------%
\mainmatter

% Alle Abkürzungen zurücksetzen, damit Abkürzungen, die im Abstract verwendet wurden
% nochmal formal eingeführt werden.
\acresetall

%Literatur einfügen     \cite[S.3]{bibkey} 
% In JabRef die zu zitierende Quelle auswählen und STRG+L (z.B. bei Verwendun von TeXstudio) oder STRG+K (hier wird direkt "\cite{bibkey}" kopiert) drücken. Dann mit STRG+V an der Stelle einfügen, an welcher man den Literaturverweis einfügen möchte.
%\section*{Spervermerk} Nicht sichtbares Kapitel in der Gliederung

% In TeXstudio
% Strg+T  Kommentiert Abschnitte oder Zeilen aus 
% Strg+U  Entfernt die auskommentierung


\chapter{Einleitung (Alex 600)}
\label{sec:Einleitung} %Textmarke/Positionsmarke, um mit Autoref darauf zu verweisen.

\todotext{Quellen von Word nach Latex übertragen}

Die zunehmende Komplexität von IT-Systemen, insbesondere durch den Übergang zu Microservice-Architekturen, stellt Unternehmen vor neue Herausforderungen bei der Verwaltung und Fehlerbehebung dieser Systeme . Microservices bestehen aus kleinen, unabhängigen Services, die gemeinsam eine größere Anwendung bilden . Diese Architektur bietet Vorteile wie eine verbesserte Skalierbarkeit und Flexibilität . Allerdings führt die verteilte Natur von Microservices zu erheblichen Schwierigkeiten bei der Identifikation von Fehlerursachen, was die Root-Cause-Analyse (RCA) besonders herausfordernd macht .

Eine zuverlässige und schnelle RCA ist entscheidend, um technische Fehler und Anomalien in IT-Systemen frühzeitig zu finden und zu beheben. Herkömmliche Ansätze zur RCA basieren oft auf der manuellen Analyse von Logs und Metriken, was bei großen, verteilten Systemen zeitaufwendig und fehleranfällig ist . Mit der Einführung von Frameworks wie Nezha ist es möglich, den RCA-Prozess zu automatisieren und feingranulare Fehlerursachen auf Code-Ebene oder Ressourcentyp-Ebene zu identifizieren, indem multi-modale Beobachtungsdaten wie Metriken, Traces und Logs verwendet werden. Nezha transformiert diese unterschiedlichen Daten in eine einheitliche Darstellung, extrahiert daraus ein Muster und ermöglicht so eine interpretierbare RCA mit hoher Genauigkeit .


\section{Problemstellung}

Trotz dieser Fortschritte gibt es weiterhin Verbesserungspotenzial, insbesondere bei der Interpretation und Priorisierung von Fehlerursachen. Hier setzt diese Arbeit an: Es wird untersucht, ob die Integration eines Large Language Models (LLM), das auf der Verarbeitung natürlicher Sprache trainiert wurde, die Ergebnisse von Nezha weiter verbessern kann. Ziel ist es, die von Nezha generierten Ranglisten der Fehlerursachen als Eingabedaten für das LLM zu verwenden und zu überprüfen, ob das Modell in der Lage ist, präzisere oder ergänzende Einschätzungen zur tatsächlichen Fehlerursache abzugeben.

\section{Lösungsansatz}
\label{sec:loesungsansatz}

Nezha erreicht bereits eine Genauigkeit von bis zu 89,77 Prozent bei der Klassifizierung der Fehlerursache, indem es multi-modale Daten kombiniert und Ereignisgraphen erstellt, die fehlerfreien und fehlerhaften Phasen von Microservices vergleichen . Diese Ranglisten der verdächtigen Ursachen bieten die Grundlage für das LLM, das durch seine Fähigkeit zur Analyse sprachlicher Muster potenziell zusätzliche Einsichten liefern kann. Insbesondere wird geprüft, ob das LLM durch die Analyse, der von Nezha gelieferten Daten eine Einschätzung der Fehlerursache abgeben und somit die Genauigkeit der RCA weiter steigern kann.

Die zentrale Forschungsfrage dieser Arbeit lautet daher: Kann ein LLM auf Grundlage, der von Nezha gelieferten Daten zur Fehleranalyse eine fundierte Einschätzung über die tatsächliche Fehlerursache abgeben und somit die Genauigkeit der RCA in Microservices steigern?

Die Arbeit verfolgt das Ziel, durch die Integration des LLMs die bestehenden Mechanismen zur Fehleridentifikation in Microservices zu optimieren und die Effizienz der RCA des Nezha Frameworks zu verbessern.


\section{Aufbau der Arbeit}
%In \autoref{sec:Kapitel2} wird das und das Thema\todo{Bla bla}{}  behandelt...
%Dieses und jenes Thema wird in \autoref{sec:Kapitel3} näher betrachtet...

%\blindtext



 

\chapter{Theorie (Domänenverständnis) 900 (Alex)}
\label{sec:Theorie}

Die Verfügbarkeit IT-Infrastrukturen ist einer der wichtigsten Kennzahlen für Unternehmen geworden. Insbesondere im Zeitalter von Cloud-basierten Microservice-Architekturen ist es der Standard, stark verteilte Systemen zu entwickeln, die es ermöglichen, skalierbare und flexible Anwendungen bereitzustellen . Diese Microservices bieten zahlreiche Vorteile gegenüber monolithischen Architekturen, darunter verbesserte Flexibilität, leichtere Wartbarkeit und die Möglichkeit, unabhängige Services ohne Ausfallzeiten zu aktualisieren . Allerdings erhöhen sie auch die Komplexität, insbesondere in Bezug auf die Fehleranalyse und das Finden der eigentlichen Fehlerursache.

\section{RCA in Microservice-Architekturen}
\label{sec:RCA in Microservice-Architekturen}

Microservices interagieren auf unterschiedliche Weise miteinander und sind oft auf verschiedene Hosts verteilt. Diese Verteilung erschwert die Diagnose von Fehlern, da sich Performance-Probleme, Systemausfälle oder Betriebsanomalien über mehrere Services hinweg ausbreiten können. Wenn in einem Microservice-Cluster ein Fehler auftritt, wird häufig eine große Menge an (multi-modalen) Daten erzeugt, darunter Logs, Metriken und Traces, die analysiert werden müssen, um die eigentliche Ursache des Problems zu identifizieren .


\section{Integration eines LLM: Verbesserung der RCA}
%
%	z.B. Text des zweiten Autors
%

\section{Nezha: Ein Framework für RCA}


\section{Nezha’s Pattern Ranking}


\section{Weitere Kommandos}
Dieser Abschnitt befasst sich mit weiteren nützlichen \LaTeX-Kommandos.

\subsection{Literaturzitate}
\label{sec:literaturzitate}

Im Lehrbuch \autocite{Schukat-Talamazzini1995}
finden sich Hinweise auf einschlägige Verfahren der automatischen Spracherkennung.

Das Literaturverwaltungsprogramm JabRef \autocite{Kopp2018} ist für viele Plattformen verfügbar und unterstützt bei der Literaturrecherche. Es ist prädestiniert dazu, mit {\LaTeX} in Kombination mit {Bib\TeX} zusammenzuarbeiten.

Über die Literaturrecherche haben Sie Zugriff auf das Buch "`Das Textverarbeitungssystem LaTeX"' \autocite[S. 15]{Oechsner2015}. Hierzu können Sie auch direkt den DOI-Link im Literaturverzeichnis anklicken.

\citeauthor{TyrinMultiAgent2012} beschreiben in \autocite{TyrinMultiAgent2012} etwas über Multi-Agenten Systeme. 

\citeauthor{Ghazali2012} beschreiben in \autocite{Ghazali2012} eine Gleichung aus \autocite{moore2007basic}.

Ein Experteninterview wird als Quelle verwendet. \citeauthor{Interviewpartnerin2021} hat gesagt, dass... \autocite[Aussage-Nr]{Interviewpartnerin2021}

In \autocite{Mamache2022} beschreiben \citeauthor{Mamache2022} dies und das...

Im aktuellen Artikel \autocite{Schaible2023} wird etwas beschrieben...

In ihrem Artikel \autocite[S. 18]{Oechsner2015} schreiben \citeauthor{Oechsner2015} dies und das.
% \autocite[Vgl.][S. 18]{Oechsner2015} // Vgl. nur bei nicht-IEEE...


\subsection{Einbinden von Bildern}

Sie können mit der Anweisung \lstinline|\includegraphics{datei}| eine Grafikdatei einbinden. Diese kann im PDF-, JPG- oder PNG-Format vorliegen. Grafiken werden üblicherweise in die float-Umgebung \lstinline|figure| gekapselt. Der Assistent in TeXstudio \autocite{vanderZander2018} tut dies automatisch, wenn das Verhalten nicht explizit abgeschaltet wird. \textit{Jede eingebundene Grafik muss vom Text aus referenziert werden.} Die textuelle Referenz hat \textit{vor} der Grafik zu erfolgen. Ein eingebundenes PDF ist in \autoref{fig:AufbauMustererkennungssystem} zu sehen. Eine weitere Grafik ist in \autoref{fig:my_label}.

\begin{figure}
\centering
\includegraphics [width=0.5\linewidth] {bilder/bildchen.pdf}
\caption [Aufbau eines Mustererkennungssystems]{Aufbau eines Mustererkennungssystems und das Beispiel einer sehr langen Bildbeschreibung, die sehr ausführlich auf die Inhalte des Bildes zu sprechen kommt. \autocite{Oechsner2015}}
\label{fig:AufbauMustererkennungssystem}
\end{figure}

\begin{figure}
    \centering
    \includegraphics[width=0.5\linewidth] {bilder/GrafikEinfuegen.png}
    \caption{Hier die Bildunterschrift}
    \label{fig:my_label}
\end{figure}

Auch andere Grafikformate werden unterstützt. Verwenden Sie den Assistenten in TeXstudio, um komfortabel Grafiken einzufügen. Siehe dazu \autoref{fig:GrafikEinfuegen}. Die Grafiken finden sich nicht notwendigerweise direkt am Einfüge-Ort. Das Textsatzsystem richtet es so ein, dass es gut aussieht.

\begin{figure}
\centering
\includegraphics[width=0.7\linewidth]{bilder/GrafikEinfuegen}
\caption{Verwendung des Assistenten in \TeX studio, um eine Grafik einzufügen.}
\label{fig:GrafikEinfuegen}
\end{figure}

\subsection{Mehrere Bilder in einer Abbildung}

Manchmal ist es notwendig, mehrere Bilder in eine einzige Abbildung zu packen. Dazu kann man die \lstinline|subfigure|-Umgebung nutzen. Ein Beispiel dafür ist in \autoref{fig:subfigures} mit den Unterbildern \autoref{fig:subfig_1} und \autoref{fig:subfig_2} zu sehen. Der dazugehörige Code ist in der Vorlage und in \autoref{lst:subfigure} zu finden.

\begin{figure}
    \centering
    \begin{subfigure}{0.45\textwidth}
        \centering
        \missingfigure[figwidth=\textwidth]{Missing}
        \caption{Unterbild 1}
        \label{fig:subfig_1}
    \end{subfigure}
    \hfill
    \begin{subfigure}{0.45\textwidth}
        \centering
        \missingfigure[figwidth=\textwidth]{Missing}
        \caption{Unterbild 2}
        \label{fig:subfig_2}
    \end{subfigure}
    \caption{Gesamtbeschreibung der Unterbilder}
    \label{fig:subfigures}
\end{figure}

\begin{lstlisting}[language=TeX, float, caption={\LaTeX-Code für subfigures}, label={lst:subfigure}]
\begin{figure}
    \centering
    \begin{subfigure}{0.45\textwidth}
        \centering
        \missingfigure[figwidth=\textwidth]{Missing}
        \caption{Unterbild 1}
        \label{fig:subfig_1}
    \end{subfigure}
    \hfill
    \begin{subfigure}{0.45\textwidth}
        \centering
        \missingfigure[figwidth=\textwidth]{Missing}
        \caption{Unterbild 2}
        \label{fig:subfig_2}
    \end{subfigure}
    \caption{Gesamtbeschreibung der Unterbilder}
    \label{fig:subfigures}
\end{figure}
\end{lstlisting}

\section{Einbinden von Codeausschnitten}
Zur Verdeutlichung von Programmabläufen kann es hilfreich sein, Codeausschnitte darzustellen. In \LaTeX geht das über die \lstinline|lstlisting|-Umgebung des \lstinline|listings| Pakets:
\lstinputlisting[language={[LaTeX]TeX},nolol]{latex/lstlisting.tex}
\begin{lstlisting}[language=Python, float, caption={Simple Python program}, label={lst:python}]
if __name__ == "__main__":
    print("This is a listing")
\end{lstlisting}
Die Ausgabe des gezeigten Codeabschnitts ist in \autoref{lst:python} zu sehen.

\section{Abkürzungen}
\label{sec:Abkuerzungen}
Abkürzungen werden mit \lstinline|\ac{...}| in den Text geschrieben. Beispiel der ersten Verwendung: \ac{JSON}
Und ab der zweiten Verwendung: \ac{JSON}

Generell gilt: Abkürzungen nur dann verwenden, wenn dadurch die \emph{Lesbarkeit erhöht} wird.


\subsection{Querverweise}\label{sec:querverweise}
Verwenden Sie unter TeXstudio die rechte Maustaste in der Strukturübersicht, um für einen Abschnitt ein Label zu erzeugen, auf das Sie Bezug nehmen können (siehe \autoref{fig:LabelErzeugen}).

\begin{figure}
\centering
\includegraphics[width=0.4\linewidth]{bilder/LabelErzeugen}
\caption{Automatische Erzeugung eines Labels für Querverweise}
\label{fig:LabelErzeugen}
\end{figure}

Es wird ein Eintrag \verb|\label{key}| erzeugt, auf den man beispielsweise mit
\verb|\autoref{key}| verweisen kann. Verwendet man \verb|\autoref|, wird der Typ des Objekts
(z.\,B. Abbildung, Tabelle, etc.) mit ausgegeben. Verwendet man nur \verb|\ref|,
so wird nur die Nummerierung des Objekts ausgegeben. In den Anhang kann man darüber genauso verweisen: \autoref{sec:ThemaDesErstenAnhangs}, \autoref{fig:my_label_im_anhang}.

\subsection{Erstellen von Tabellen}

Das Volk hat gesprochen. Siehe \autoref{tab:tabular}. Auch hier kommt ein float zum Einsatz, jedoch mit dem Positionierungs-Hinweis \verb|[h]|. Jedes float sollte mit einem Label versehen werden, und es sollte im Text darauf verwiesen werden, da sich die Position ändern kann.

\begin{table}[h]
\centering
\begin{tabular} {|llc||r|}
	\hline
	Name & Rang & Fraktion & Stimmenanteil \\
	\hline
	Mobutu & General & CDU & 57\% \\
	Tsvangirai & Oberst & CSU & 63\% \\
	\hline
\end{tabular}
\caption {Bundestagswahl in Simbabwe}
\label{tab:tabular}
\end{table}

\autoref{tab:booktab} verwendet keine vertikalen Linien und entspricht dem üblicherweise in Büchern verwenden Stil. Derartige Tabellen sind deutlich ansehnlicher.

\begin{table}[h]
\centering
\begin{tabular} {llcr}
	\toprule
	Name & Rang & Fraktion & Stimmenanteil \\
	\midrule
	Mobutu & General & CDU & 57\% \\
	Tsvangirai & Oberst & CSU & 63\% \\
	\bottomrule
\end{tabular}
\caption {Bundestagswahl in Simbabwe}
\label{tab:booktab}
\end{table}

\subsection{Listen und Aufzählungen}

Listen und Aufzählungen werden in einer Umgebung angelegt (umschlossen von \verb|begin| und \verb|end|.). \verb|\begin{itemize}| leitet eine Liste ein und \verb|\begin{enumerate}| eine Aufzählung. Die Einträge werden jeweils mit \verb|\item| begonnen. Folgend zwei Beispiele.

Itemize:
\begin{itemize}
\item Test
\item Test
\item Test
\end{itemize}

Enumerate:
\begin{enumerate}
\item Test
\item Test
\item Test
\end{enumerate}

\subsection{Mathematische Formeln}

Mathematische Formeln werden mittels \verb|\(...\)|
in den Fließtext eingebaut
--- zum Beispiel \( E=mc^2 \) und:
sei \(V\) ein Vektorraum über \(\mathbb{R}\)
und \(\mathcal{M}\) eine Indexmenge
---
oder aber mittels \verb|\[...\]|
abgesetzt und zentriert dargestellt:
	\[
	\pmb{x} = \sqrt[3]{\frac{a^2-b^2}{a^2+b^2}}
		~~~~~ \text{versus} ~~~~~
	\boldsymbol{x} = \sqrt[3]{\frac{a^2-b^2}{a^2+b^2}}
	\]


\subsection{Silbentrennung}
Vertrauen Sie bitte nie einer automatischen Silbentrennung (auch nicht der von Microsoft Word \& Co.). In folgendem Test-Absatz ist das Wort "`Spracherkennung"' falsch getrennt.

Testzeile Testzeile Testzeile Testzeile Testzeile Testzeile Testzeile Testzeile Testzei Spracherkennung.

Sie können LaTeX die richtige Trennung mit \verb|\hyphenation{...}| mitteilen. Man tut das üblicherweise noch vor \verb|\begin{document}|.

\hyphenation{Sprach-er-ken-nung}	% hier jetzt zu Demonstrationszwecken im Text
Testzeile Testzeile Testzeile Testzeile Testzeile Testzeile Testzeile Testzeile Testzei Spracherkennung.



\section{Abschnitt im Kapitel zwei...}

\chapter{Datenverständnis Inhalte...(Paul 800)}
\label{sec:Kapitel3}

Hier kommt eine Übersicht des Datenverständnis Kapitels hin.

\section{Abschnitt im Kapitel Datenverständnis...}
\label{sec:abschnitt3.1}

\section{Abschnitt im Kapitel Datenverständnis...}

\section{Abschnitt im Kapitel Datenverständnis...}
\chapter{Datenvorverarbeitung Inhalte (Paul 700)}
\label{sec:Kapitel3}

Hier kommt eine Übersicht des Datenvorverarbeitung Kapitels hin.

\section{Abschnitt im Kapitel Datenvorverarbeitung...}
\label{sec:abschnitt3.1}

\section{Abschnitt im Kapitel Datenvorverarbeitung...}

\section{Abschnitt im Kapitel Datenvorverarbeitung...}
\chapter{Modelierung 1000 (Henning)}
\label{sec:Kapitel3}

Dieses Kapitel thematisiert die Phase Modeling des ref CRISP-DM Prozesses. Diese Phase hat eine hohe relevanz und wird typischerweise bei Projekten welche CRISP-DM methodisch verwenden mehrfach iteriert quelle. Ziel dieser Phase ist es aus den zuvor vorbereiteten Daten und gewonnen Erkenntnissen einen Mehrwert zu stifen quelle. Im Konkreten Fall sollen die zuvor gesammelten Daten genutzt werden um ein LLM acro in die Lage zu versetzen eine Schätzung zu machen welcher RCA acro quelle vorliegt. Innerhalb des Kapitels werden Ansätze zur Lösung diskutiert, diese sind die Zusammenfassung der verschiedenen Iterationen des durchlaufenen CRISP-dm acro Prozesses. Ebenso werden besondere Herausforderungen sowie erste Erkenntnisse dargestellt, bevor dann im folgenden Kapitel die Qualität der Antworten der verwendeten LLMs qualitativ bewertet werden kann. 

\section{Zielsetzung}
\label{sec:Zielsetzung}

Die Ausgaben aus dem modell sollen nicht nur die Forschungsfrage beantworten (quelle todo src) sondern auch mit den von nezha (src ref) gelieferten Ergebnissen verglichen werden. Da die Ausgabe von nezha aus einer Liste von möglichen RCA auswählt entspricht die durch das LLM zu bewerkstelligende Aufgabe am ehestesten der einer Klassifikation. Nezha bezeichnet die Klassen als:

Liste, nzeha name einfügen quelle:

  - exception
  - return
  - cpu contemtion
  - network
  - x

  Der zu definierende Prompt muss demnach ebenso zu einem sogenannten exact match (EM) führen, die spätere Validierung und Evaluierung des Verfahrens und der Ausgaben muss diese Maßgabe berücksichtigen. Da die bloße Einteilung in diese wenigen Klassen jedoch noch nicht den maximalen Nutzen stiftet wäre grundsätzlich die Erweiterung der Ausgaben über die Klassifizierung hinweg sehr positiv bzw. ach nötig. Dazu wären im Falle von Codebugs (Exceptions und returns) eine Angabe zur Codestelle oder ein Trace denkbar, im Falle von Netzwerk oder CPU-Fehlern eine genaue Angabe des betroffnen PODs, Dienstes oder Hosts.

\section{Datenaufteilung}
\label{sec:datasplit}
Ein normaler Ansatz wäre ein vortrainiertes LLM noch nachzutrainieren tbd rs quelle. Um dieses Nachtraining durchzuführen ist eine Aufteilung der bestehenden Daten in Trainingsdaten, Validierungsdaten und Testdaten oft sinnhaft qquelle. Konkret ist das in den verwendete Daten set tbd aus nezha ist frei verfügbar, es beinhaltet in der bereinigten, vorverarbeiteten Version ~112 Records, dem liegen 56 injezierte Fehler zugrunde. Typisch ist eine herangehensweise mit einer 70-20-10 Aufteilung, wobei die Aufteilung mit 70\% Trainingsdaten, 10\% Validierungsdaten und  20\% Testdaten (prüfen quelle) für viele Anwendungsfälle passend ist. Im vorliegenden konkreten Fall ist jedoch Wie bereits in  (ref Understanding) dargestellt, leider nur ein vergleichsweise geringer Datenbestand verfügbar. Ein Training auf diesen Bestand nach der vorher gezeigten Aufteilung ist als wenig stabil zu bewerten (quelle). Um eine Art Datenaufteilung dennoch auszuführen wurden im Rahmen der Versuche zum Modeling die Verfahren Few-shot und One-shot angewandt. Beide basieren darauf [Erklärtext, ggf. ref theorie] mit weniger bzw., fast keinen Daten ein Retraining durchzuführen. Im Ergebnis konnten die besten Resultate erzielt werden wenn kein Nachtraining erfolgte. Dieses Vorgehen wird auch als Zero-Shot Prompting bezeichnet, nähere Ausführungen dazu unter \autoref{sec:prompte}


Da dieser Datenbestand zu klein ist um ein sinnhaftes Training für ein LLM durchzuführen quelle, musste eine andere 

\section{Prompt-eng.}
\label{sec:prompte}
Wie zuvor dargestellt sollen die vorverarbeiteten Daten an ein LLM gegeben werden (ref). Dies geschieht typsischerweise mittels eines sogenannten Prompts, dies ist in den meisten Fällen eine textbasierte eingabe (quelle). Im konkreten Fall soll diese Eingabe nicht interaktiv sondern mittels API (acro) durch ein Programm erfolgen. Um die gemachten Eingaben zu optimieren bedarf es Methoden und Instrumente des Prompt-engineering. Ziel ist, die Prompts so zu gestalten, das möglichst nützliche, spezifische und qualitativ verwendare Ausgaben durch das LLM generiert werden. Ein Blick in die Literatur offenbart eine Vielzahl von möglichen Techniken zu Promptgestaltung und optimierung. quelle Listet etwa 29 Techniken, kategoriesiert in 12 Anwendungsbereiche\autocite[S. 3]{Sahoo2024 S.3}. Die Techniken folgen in der Regel gewisser Methodik und Denkmustern, sie schließen sich teilweise auch auch aus, es galt also zum konkreten Anwendungsfall passende Techniken auszuwählen. Dem Thread of Thought (ThoT) acro Ansatz, beschrieben in quelle \autoref{Zhou2023}, wurde sich dabei angenähert. Dieser Ansatz beschreibt ein zunächst chaotischen Kontext schritt weise zu nähern, iterativ zu verfeinern und die Ergebnisse mittels sogenanntem Exact Match (quelle acro em) zu messen. Thot zeigt in Experimenten das es ein verbessertes Schlussfolgern aufweist, dies ist insbesondere im hier vorliegenden Anwendungsfall, dem finden einer ursprünglichen Fehlerursache von essentieller Bedeutung. Bereits nach wenigen Versuchen konnte programmatisch ein verwertbares Ergebnis erzielt werden, dies spricht zum einen für eine leichte Handhabbarkeit von LLMs (acro) via python über API als auch über die Wirkungsmächtigkeit der angwandten Methodik über Thot (acro). Mehrere Techniken wie x und y quelle, bedingen z.B. ein komplexes Retraining, welches wie bereits in \autoref{sec:datasplit} bedingt durch die geringe Datenmenge kaum sinnhaft möglich war.



Folgt man den Ausführungen zum Prompt-engineering von quelle sind dazu folgende Punkte beachtenswert:

Few-Shot Prompting: Das Modell erhält wenige Beispiele innerhalb des Prompts, um die gewünschte Aufgabe zu verstehen und auszuführen. Dies verbessert die Genauigkeit, indem es dem Modell Kontext und Struktur für die erwartete Antwort bietet. 

% Henning + Alex (1000)

\chapter{Evaluierung}
\label{cha:eval}

Dieses Kapitel beschäftigt sich, \ac{CRISP-DM} folgend, mit der Evaluierung des zuvor erarbeiteten und in \autoref{cha:Modellierung} beschriebenen Modells. Dabei ist im Kern die wichtigste Komponente der Prompt, da dieser neben den vorverarbeiteten Daten die relevanteste Stellschraube ist um das Ergebnis zu beeinflussen. Zwar wurden auch mit verschiedene LLMs interagiert, dies deinte aber mehr der generlisierung des Ansatzes, weniger der Vergleich der verschiednenden \ac{LLM}-Varianten.

Während interaktiv noch gut festgestellt werden kann, ob die Antwortqualität eines \ac{LLM} eher zunimmt oder abnimmt ist dies bei der automatisierten, programmatischen Nutzung schwieriger. Im vorliegenden Fall erfolgt die Nutzung des \ac{LLM}s wie unter \autoref{sec:prompte} beschrieben mittels \ac{API}. Die Idee war daher, auch die Ergebnisse programmatisch zu überprüfen.



Kernidee: Erkläre warum evaluiert wird

Wie evaluiert wird

Welche Technik und Literatur zum evaluieren verwendet wird.

Wie die Ergebnisse aussehen, wo "gut" klappte, was noch besser sein könnte. Ggf. Was richtung "Learning/Conclusion". Einhaltung CRISP-DM zeigen und aufs iterative referenzieren und Rücksprung auf die ersten Kapitel machen.

Hier kommt eine Übersicht des Evaluierung Kapitels hin.

\section{Test - Framework}
\label{sec:Framework}
Um programmatische Tests des Prompts durchzuführen und zu bewerten wurde das Tool promptfoo (quelle ref.) verwendet. Dieses Framework ist dienlich bei der Enwiklung, Verbesserung und dem Test von \ac{LLM}s und Prompts. Dem Framework wurden folgende Inputs gegeben:

\begin{itemize}
    \item promptfooconfig.yaml, konfiguriert den test grundsätzlich, definiert promptquelle, die provider und die Datenquelle
    \item prompts.txt, definiert den prompt, exemplarisch in \autoref{fig:finalprompt} zu sehen
    \item tests.csv liefert Datenquellen
\end{itemize}

Der Dokumentation (quelle) von promptfoo für die Deklaration von csv-Files ist ensprechend gefolgt worden\footnote{Relevantestes Merkmal ist die Spalte \_\_expected}. Mit den so vorbereiteten Daten ist promptfoo in der Lage das Ergebnis des \ac{LLM}s mit der wirklichen Fehlerursache abzugleichen und darauf basierend eine Aussage zur Genauigkeit zu treffen. Die Spalte \_\_expected übernimmt dabei die Funktion eines Asserts, also einer Behauptung die es zu prüfen gilt. Durch die Nutzung von Asserts lässt sich die Erwartung des Ergebnisses hinterlegen und in der Folge dann abgleichen. 

\section{Methodik}
Sobald das Testframework wie unter \autoref{sec:framework} konfiguriert war, konnten die Tests programmatisch durchlaufen werden. Bei Änderungen des Prompts und einem erneuten Testdurchlauf konnte direkt abgelesen werden ob die Trefferrate gestiegen oder gefallen ist. Eine beispielhafte Ausgabe von promptfoo während der Entwicklung ist in \autoref{} zu sehen. Neben der Trefferrate\footnote{xx\% passing (x/y cases)} werden auch der verwendete Prompt als auch Metadaten zum \ac{API}-Aufruf ersichtlich. Zum Beispiel die Latenz und die Kosten des Aufrufes in Euro. Hier wird die Stärke des iterativen Vorgehensmodells nach \ac{CRISP-DM} deutlich, denn nach jeder Änderung am Prompt oder auch an den angelieferten Daten kann die Veränderung des Ergebnisses direkt abgelesen werden. Ebenso werden verschiedene \ac{LLM} mit den identischen Eingabeparametern versorgt und die Ergebnisse erlauben einen Quervergleich der einzelnen \ac{LLM}s untereinander. Neben den abgebildeten Modellen von OpenAI wurden auch lokale LLMs\footnote{Z.B. Llama-3.3-70B-Instruct-GGUF mittels LM Studio} in die Tests miteinbezogen. Die Ergebenisse der lokalen Modelle blieben weit unter den Erwartungen zurück, Trefferraten unter 50%.

\section{Ergebnisse}
Als Baseline zur Erkennung der RCA (quelle ac) ggf. ref Kapitel vorher und Foschungsfrage!! dient die Angabe der Fehlerursache aus dem nezha-Framework. Nach mehreren iterationen der Promptentwicklung und Änderung der priorisierung der error logs sowie ergänzung 

4o-mini: 96.43% passing (54/56 cases)
4o: 98.21% passing (55/56 cases)

Als Schlussfolgerund der durchgeführten Evaluierung kann festgehalten werden, das zum einen LLMs sehr gut in der Lage sind, ...


\chapter{Zusammenfassung und Ausblick (alle 1000)}
\label{sec:ZusammenfassungUndAusblick}

\section{Fazit}
\label{sec:Fazit}

Ein Fazit

\section{Ausblick}
\label{sec:Ausblick}

Ein Ausblick...






%-----------------------------------------------------------------------------%
% Anhang                                                                      %
%-----------------------------------------------------------------------------%
\appendix

\chapter{Thema des ersten Anhangs}
\label{sec:ThemaDesErstenAnhangs}
Die Überschrift ist echt blöd gewählt...
\section{Der erste Abschnitt des ersten Anhangs}
Nach einer Überschrift kommt bekanntlich immer Text.

\begin{figure}
    \centering
    \includegraphics[width=1\linewidth]{bilder/statistic_id2984_fahrleistung-von-pkw-in-deutschland-bis-2019.pdf}
    \caption{Caption}
    \label{fig:my_label_im_anhang}
\end{figure}


\chapter{Experteninterview: \citeauthor{Interviewpartnerin2021}}

\begin{longtable}{lp{5cm}p{5cm}c}
\caption{Ein Beispiel f{\"u}r Longtable} \\
\label{tab:my_label_im_anhang}
% Definition des Tabellenkopfes auf der ersten Seite
Nr & Transkript & Zusammenfassung  & Kategorie\\
\toprule
\endfirsthead % Erster Kopf zu Ende
% Definition des Tabellenkopfes auf den folgenden Seiten
\caption{Lange Tabelle mit Logtable Fortsetzung}\\
Nr & Transkript & Zusammenfassung  & Kategorie\\
\toprule
\endhead % Zweiter Kopf ist zu Ende
\midrule
\multicolumn{4}{r}{Vor dem endfoot Weiter auf der n{\"a}chste Seite}\\
\endfoot
\bottomrule
\multicolumn{4}{r}{Vor dem endlastfoot Tabelle zu Ende} \\
\endlastfoot
% Ab hier kommt der Inhalt der Tabelle
    1     & Eine tolle und wichtige Aussage & Aussage & H1.1\\
    1     & Eine tolle und wichtige Aussage & Aussage & H1.1\\
    1     & Eine tolle und wichtige Aussage & Aussage & H1.1\\
    1     & Eine tolle und wichtige Aussage & Aussage & H1.1\\
    1     & Eine tolle und wichtige Aussage & Aussage & H1.1\\
    1     & Eine tolle und wichtige Aussage & Aussage & H1.1\\
    1     & Eine tolle und wichtige Aussage & Aussage & H1.1\\
    1     & Eine tolle und wichtige Aussage & Aussage & H1.1\\
    1     & Eine tolle und wichtige Aussage & Aussage & H1.1\\
    1     & Eine tolle und wichtige Aussage & Aussage & H1.1\\
    1     & Eine tolle und wichtige Aussage & Aussage & H1.1\\
    1     & Eine tolle und wichtige Aussage & Aussage & H1.1\\
    1     & Eine tolle und wichtige Aussage & Aussage & H1.1\\
    1     & Eine tolle und wichtige Aussage & Aussage & H1.1\\
    1     & Eine tolle und wichtige Aussage & Aussage & H1.1\\
    1     & Eine tolle und wichtige Aussage & Aussage & H1.1\\
    1     & Eine tolle und wichtige Aussage & Aussage & H1.1\\
    1     & Eine tolle und wichtige Aussage & Aussage & H1.1\\
    1     & Eine tolle und wichtige Aussage & Aussage & H1.1\\
    1     & Eine tolle und wichtige Aussage & Aussage & H1.1\\
    1     & Eine tolle und wichtige Aussage & Aussage & H1.1\\
    1     & Eine tolle und wichtige Aussage & Aussage & H1.1\\
    1     & Eine tolle und wichtige Aussage & Aussage & H1.1\\
    1     & Eine tolle und wichtige Aussage & Aussage & H1.1\\
    1     & Eine tolle und wichtige Aussage & Aussage & H1.1\\
    1     & Eine tolle und wichtige Aussage & Aussage & H1.1\\
    1     & Eine tolle und wichtige Aussage & Aussage & H1.1\\
    1     & Eine tolle und wichtige Aussage & Aussage & H1.1\\
    1     & Eine tolle und wichtige Aussage & Aussage & H1.1\\
    1     & Eine tolle und wichtige Aussage & Aussage & H1.1\\
    1     & Eine tolle und wichtige Aussage & Aussage & H1.1\\
    1     & Eine tolle und wichtige Aussage & Aussage & H1.1\\
    1     & Eine tolle und wichtige Aussage & Aussage & H1.1\\
    1     & Eine tolle und wichtige Aussage & Aussage & H1.1\\
    1     & Eine tolle und wichtige Aussage & Aussage & H1.1\\
    1     & Eine tolle und wichtige Aussage & Aussage & H1.1\\
    1     & Eine tolle und wichtige Aussage & Aussage & H1.1\\
\end{longtable}

\backmatter
%-----------------------------------------------------------------------------%
% Literaturverzeichnis                                                        %
%-----------------------------------------------------------------------------%
\cleardoublepage
% \printbibliography
% Hier kann das Literaturverzeichnis noch getrennt werden.
% Dafür ist zwingend biblatex notwendig!
\ohead[]{Literatur}
\printbibliography[nottype=misc]

\printbibliography[type=misc,notkeyword=Experteninterview,title={Sonstige Quellen}]

\printbibliography[type=misc,keyword=Experteninterview,title={Experteninterviews}]

\cleardoublepage
\ohead[]{\headmark}
\input{latex/ee}


\end{document}